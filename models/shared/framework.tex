%% framework.tex - Discrete Choice Model Framework for Isolated Validation Models
%% Focused documentation of the six models in the models/ folder
%% For inclusion in main document via %% framework.tex - Discrete Choice Model Framework for Isolated Validation Models
%% Focused documentation of the six models in the models/ folder
%% For inclusion in main document via %% framework.tex - Discrete Choice Model Framework for Isolated Validation Models
%% Focused documentation of the six models in the models/ folder
%% For inclusion in main document via %% framework.tex - Discrete Choice Model Framework for Isolated Validation Models
%% Focused documentation of the six models in the models/ folder
%% For inclusion in main document via \input{framework.tex}
%%
%% Required packages in main document:
%%   \usepackage{amsmath}
%%   \usepackage{amssymb}
%%   \usepackage{booktabs}
%%   \usepackage{threeparttable}
%%   \usepackage{multirow}

\section{Discrete Choice Model Framework}
\label{sec:dcm_framework}

This section presents the discrete choice modeling framework employed in our empirical analysis. We implement a sequence of six model specifications, each building upon its predecessor to address increasingly sophisticated forms of preference heterogeneity. The progression from basic multinomial logit through integrated choice and latent variable models provides both methodological rigor and substantive insights into the determinants of choice behavior.

%% ============================================================================
%% SUBSECTION: Framework Overview
%% ============================================================================
\subsection{Framework Overview}
\label{sec:framework_overview}

Our modeling framework comprises six discrete choice specifications organized in a deliberate hierarchy of complexity. Each model addresses a specific limitation of its predecessor while maintaining consistency in core structural assumptions. Table~\ref{tab:model_hierarchy} summarizes this progression.

\begin{table}[htbp]
\centering
\caption{Model Hierarchy and Heterogeneity Treatment}
\label{tab:model_hierarchy}
\begin{threeparttable}
\begin{tabular}{llccl}
\toprule
Model & Type & $K$ & Heterogeneity & Key Innovation \\
\midrule
1. MNL Basic & Baseline & 3 & None & Foundation specification \\
2. MNL Demographics & Observable & 8 & Demographics & Taste-demographic interactions \\
3. MXL Basic & Random Coef. & 4 & Unobserved & Distributional heterogeneity \\
4. HCM Basic & Hybrid & 5+ & Latent (single) & Psychological constructs \\
5. HCM Full & Hybrid & 11+ & Latent (multiple) & Domain-specific effects \\
6. ICLV & Integrated & 10+ & Latent (simult.) & Bias elimination \\
\bottomrule
\end{tabular}
\begin{tablenotes}
\small
\item Note: $K$ = number of estimated parameters (excluding measurement model parameters for HCM/ICLV). MNL = Multinomial Logit; MXL = Mixed Logit; HCM = Hybrid Choice Model; ICLV = Integrated Choice and Latent Variable.
\end{tablenotes}
\end{threeparttable}
\end{table}

The framework is designed around a common choice context: individuals choosing among three service alternatives---two paid expedited options and one standard (free) option with longer processing time. Attributes include processing fee (in Turkish Lira) and processing duration (in weeks). This unified context enables direct comparison of model performance and parameter estimates across specifications.

\subsubsection{Data Structure}

Each model operates on panel data with the following structure:
\begin{itemize}
\item $N = 500$ individuals
\item $T = 10$ choice tasks per individual
\item $J = 3$ alternatives per choice task
\item Total observations: $N \times T = 5{,}000$ choices
\end{itemize}

Attributes are drawn from a stratified-factorial experimental design that ensures orthogonality between fee and duration, enabling precise identification of both effects.

\subsubsection{Common Notation}

Throughout this section, we employ the following notation:
\begin{itemize}
\item $V_{nj}$ = Systematic utility of alternative $j$ for individual $n$
\item $\beta_{\text{fee}}$ = Marginal utility of fee (expected negative)
\item $\beta_{\text{dur}}$ = Marginal utility of duration (expected negative)
\item $\text{ASC}_{\text{paid}}$ = Alternative-specific constant for paid options
\item $\eta$ = Latent variable (psychological construct)
\item $\lambda$ = Factor loading (measurement model)
\item $\tau$ = Threshold parameter (ordered probit)
\end{itemize}

%% ============================================================================
%% SUBSECTION: Model 1 - MNL Basic
%% ============================================================================
\subsection{Model 1: Multinomial Logit Basic}
\label{sec:mnl_basic}

The multinomial logit (MNL) basic model serves as the foundation of our analysis, providing a parsimonious baseline against which more complex specifications are compared.

\subsubsection{Model Specification}

Under the assumption of homogeneous preferences, all individuals share identical taste parameters. The systematic utility for alternative $j$ is:
\begin{equation}
V_{nj} = \text{ASC}_{\text{paid}} \cdot \mathbf{1}_{j \in \{1,2\}} + \beta_{\text{fee}} \cdot \text{fee}_j + \beta_{\text{dur}} \cdot \text{dur}_j
\label{eq:mnl_basic_utility}
\end{equation}
where $\mathbf{1}_{j \in \{1,2\}}$ is an indicator function equal to 1 for paid alternatives and 0 for the standard option (which serves as the reference category with ASC normalized to zero).

\subsubsection{Choice Probability}

Given IID Gumbel-distributed errors, the choice probability takes the standard logit form:
\begin{equation}
P_{nj} = \frac{\exp(V_{nj})}{\exp(V_{n1}) + \exp(V_{n2}) + \exp(V_{n3})}
\label{eq:mnl_basic_prob}
\end{equation}

\subsubsection{Parameters}

The model estimates three parameters:

\begin{table}[htbp]
\centering
\caption{MNL Basic Parameters}
\label{tab:mnl_basic_params}
\begin{tabular}{llc}
\toprule
Parameter & Interpretation & Expected Sign \\
\midrule
$\text{ASC}_{\text{paid}}$ & Baseline preference for expedited service & + \\
$\beta_{\text{fee}}$ & Marginal disutility of fee (per 10,000 TL) & $-$ \\
$\beta_{\text{dur}}$ & Marginal disutility of duration (per week) & $-$ \\
\bottomrule
\end{tabular}
\end{table}

\subsubsection{Limitations}

The MNL basic model assumes that all individuals have identical preferences---a strong assumption that is unlikely to hold empirically. Observable differences in age, income, and education are ignored, as are unobserved psychological factors that may systematically influence choice.

%% ============================================================================
%% SUBSECTION: Model 2 - MNL Demographics
%% ============================================================================
\subsection{Model 2: Multinomial Logit with Demographics}
\label{sec:mnl_demo}

The MNL demographics model relaxes the homogeneity assumption by allowing taste parameters to vary as a function of observed individual characteristics.

\subsubsection{Model Specification}

We specify individual-specific coefficients that depend on demographic variables:
\begin{align}
\beta_{\text{fee},n} &= \beta_{\text{fee}} + \beta_{\text{fee,age}} \cdot \text{age}_n^c + \beta_{\text{fee,edu}} \cdot \text{edu}_n^c + \beta_{\text{fee,inc}} \cdot \text{inc}_n^c \label{eq:fee_demo} \\
\beta_{\text{dur},n} &= \beta_{\text{dur}} + \beta_{\text{dur,edu}} \cdot \text{edu}_n^c + \beta_{\text{dur,inc}} \cdot \text{inc}_n^c \label{eq:dur_demo}
\end{align}
where superscript $c$ denotes centered and scaled demographic variables.

The utility specification becomes:
\begin{equation}
V_{nj} = \text{ASC}_{\text{paid}} \cdot \mathbf{1}_{j \in \{1,2\}} + \beta_{\text{fee},n} \cdot \text{fee}_j + \beta_{\text{dur},n} \cdot \text{dur}_j
\label{eq:mnl_demo_utility}
\end{equation}

\subsubsection{Asymmetric Interaction Structure}

A notable feature of this specification is the asymmetric treatment of demographic interactions:
\begin{itemize}
\item \textbf{Fee sensitivity}: Varies with age, education, \textit{and} income
\item \textbf{Duration sensitivity}: Varies with education and income only (not age)
\end{itemize}

This asymmetry reflects theoretical considerations:
\begin{itemize}
\item \textbf{Age $\rightarrow$ Fee}: Older individuals may have accumulated wealth, reducing fee sensitivity
\item \textbf{Education/Income $\rightarrow$ Fee}: Higher socioeconomic status reduces price sensitivity
\item \textbf{Education/Income $\rightarrow$ Duration}: Higher opportunity cost of time increases duration sensitivity
\item \textbf{Age $\not\rightarrow$ Duration}: Time valuation driven by opportunity cost, not lifecycle stage
\end{itemize}

\subsubsection{Parameters}

The model estimates eight parameters:

\begin{table}[htbp]
\centering
\caption{MNL Demographics Parameters}
\label{tab:mnl_demo_params}
\begin{tabular}{llcc}
\toprule
Parameter & Interpretation & Expected Sign & Applies To \\
\midrule
$\text{ASC}_{\text{paid}}$ & Baseline expedited preference & + & All \\
$\beta_{\text{fee}}$ & Base fee sensitivity & $-$ & All \\
$\beta_{\text{fee,age}}$ & Age effect on fee sensitivity & + & Fee \\
$\beta_{\text{fee,edu}}$ & Education effect on fee sensitivity & + & Fee \\
$\beta_{\text{fee,inc}}$ & Income effect on fee sensitivity & + & Fee \\
$\beta_{\text{dur}}$ & Base duration sensitivity & $-$ & All \\
$\beta_{\text{dur,edu}}$ & Education effect on duration sensitivity & $-$ & Duration \\
$\beta_{\text{dur,inc}}$ & Income effect on duration sensitivity & $-$ & Duration \\
\bottomrule
\end{tabular}
\end{table}

Positive interaction coefficients on fee indicate \textit{reduced} fee sensitivity (less negative effective $\beta_{\text{fee}}$). Negative interaction coefficients on duration indicate \textit{increased} duration sensitivity (more negative effective $\beta_{\text{dur}}$).

%% ============================================================================
%% SUBSECTION: Model 3 - Mixed Logit
%% ============================================================================
\subsection{Model 3: Mixed Logit Basic}
\label{sec:mxl_basic}

The mixed logit (MXL) model captures unobserved preference heterogeneity by allowing taste parameters to follow continuous distributions across the population.

\subsubsection{Model Specification}

We specify the fee coefficient as normally distributed:
\begin{equation}
\beta_{\text{fee},n} \sim N(\mu_{\text{fee}}, \sigma^2_{\text{fee}})
\label{eq:mxl_distribution}
\end{equation}

Equivalently, using the standard normal draw representation:
\begin{equation}
\beta_{\text{fee},n} = \mu_{\text{fee}} + \sigma_{\text{fee}} \cdot \nu_n, \quad \nu_n \sim N(0,1)
\label{eq:mxl_draw}
\end{equation}

The utility specification is:
\begin{equation}
V_{nj} = \text{ASC}_{\text{paid}} \cdot \mathbf{1}_{j \in \{1,2\}} + \beta_{\text{fee},n} \cdot \text{fee}_j + \beta_{\text{dur}} \cdot \text{dur}_j
\label{eq:mxl_utility}
\end{equation}

Note that $\beta_{\text{dur}}$ is specified as fixed (non-random) to ensure identification and parsimony.

\subsubsection{Choice Probability}

The unconditional choice probability requires integration over the mixing distribution:
\begin{equation}
P_{nj} = \int \frac{\exp(V_{nj}(\beta))}{\sum_k \exp(V_{nk}(\beta))} \cdot \phi\left(\frac{\beta - \mu}{\sigma}\right) \cdot \frac{1}{\sigma} \, d\beta
\label{eq:mxl_prob}
\end{equation}

This integral lacks a closed-form solution. We approximate it using simulated maximum likelihood with $R = 500$ Halton draws:
\begin{equation}
\tilde{P}_{nj} = \frac{1}{R} \sum_{r=1}^{R} \frac{\exp(V_{nj}(\beta^{(r)}_n))}{\sum_k \exp(V_{nk}(\beta^{(r)}_n))}
\label{eq:mxl_simulated}
\end{equation}

\subsubsection{Parameters}

The model estimates four parameters:

\begin{table}[htbp]
\centering
\caption{MXL Basic Parameters}
\label{tab:mxl_params}
\begin{tabular}{llc}
\toprule
Parameter & Interpretation & Expected Sign \\
\midrule
$\text{ASC}_{\text{paid}}$ & Baseline expedited preference & + \\
$\mu_{\text{fee}}$ & Mean fee coefficient & $-$ \\
$\sigma_{\text{fee}}$ & Standard deviation of fee coefficient & + \\
$\beta_{\text{dur}}$ & Duration coefficient (fixed) & $-$ \\
\bottomrule
\end{tabular}
\end{table}

\subsubsection{Interpretation of Heterogeneity}

The coefficient of variation $\text{CV} = \sigma_{\text{fee}}/|\mu_{\text{fee}}|$ provides a normalized measure of heterogeneity. Under the normal specification:
\begin{itemize}
\item If $\text{CV} = 0.4$, approximately 1\% of the population has positive $\beta_{\text{fee}}$ (counterintuitive)
\item Large $\sigma$ relative to $|\mu|$ may indicate the need for alternative distributions (e.g., lognormal, constrained normal)
\end{itemize}

%% ============================================================================
%% SUBSECTION: Model 4 - HCM Basic
%% ============================================================================
\subsection{Model 4: Hybrid Choice Model Basic}
\label{sec:hcm_basic}

The hybrid choice model (HCM) introduces latent psychological constructs as determinants of choice behavior. The basic HCM specification incorporates a single latent variable.

\subsubsection{Conceptual Framework}

HCM extends the standard choice model by positing that unobserved psychological attitudes---measured imperfectly through survey indicators---systematically influence taste parameters. The model comprises three interconnected components:

\begin{enumerate}
\item \textbf{Structural model}: Links demographics to latent variables
\item \textbf{Measurement model}: Relates latent variables to observed indicators
\item \textbf{Choice model}: Incorporates latent variables as taste shifters
\end{enumerate}

\subsubsection{Structural Model}

The latent variable (Blind Patriotism, $\eta_{\text{pb}}$) is determined by demographics:
\begin{equation}
\eta_{\text{pb},n} = \gamma_0 + \gamma_{\text{age}} \cdot (\text{age}_n - \bar{\text{age}}) + \zeta_n, \quad \zeta_n \sim N(0, \sigma^2_\zeta)
\label{eq:hcm_structural}
\end{equation}

This specification reflects the hypothesis that patriotic attitudes vary systematically with age, with a residual component capturing individual-specific variation.

\subsubsection{Measurement Model}

The latent variable is measured through five Likert-scale indicators ($I_1, \ldots, I_5$), each following an ordered probit specification:
\begin{equation}
P(I_{nk} = c \,|\, \eta_n) = \Phi(\tau_{k,c} - \lambda_k \eta_n) - \Phi(\tau_{k,c-1} - \lambda_k \eta_n)
\label{eq:hcm_measurement}
\end{equation}
where:
\begin{itemize}
\item $\lambda_k$ = factor loading for indicator $k$ (first loading fixed to 1 for identification)
\item $\tau_{k,c}$ = threshold for category $c$ of indicator $k$
\item $\Phi(\cdot)$ = standard normal CDF
\end{itemize}

\subsubsection{Choice Model}

The latent variable enters the choice model as a taste shifter:
\begin{align}
\beta_{\text{fee},n} &= \beta_{\text{fee}} + \lambda_{\text{fee,pb}} \cdot \eta_{\text{pb},n} \label{eq:hcm_fee} \\
V_{nj} &= \text{ASC}_{\text{paid}} \cdot \mathbf{1}_{j \in \{1,2\}} + \beta_{\text{fee},n} \cdot \text{fee}_j + \beta_{\text{dur}} \cdot \text{dur}_j \label{eq:hcm_utility}
\end{align}

Negative $\lambda_{\text{fee,pb}}$ indicates that higher Blind Patriotism reduces fee sensitivity (individuals with stronger patriotic attitudes are more willing to pay for government services).

\subsubsection{Estimation Challenge: Attenuation Bias}

When estimated via two-stage procedures (factor analysis followed by choice model estimation), HCM suffers from attenuation bias. The estimated latent variable effect is systematically underestimated by 15--30\% due to measurement error in the first-stage latent scores.

%% ============================================================================
%% SUBSECTION: Model 5 - HCM Full
%% ============================================================================
\subsection{Model 5: Hybrid Choice Model Full}
\label{sec:hcm_full}

The full HCM specification extends the basic model to incorporate four latent psychological constructs with domain-specific effects.

\subsubsection{Four Latent Variables}

We model four attitudinal constructs:

\begin{enumerate}
\item \textbf{Blind Patriotism ($\eta_{\text{pb}}$)}: Uncritical, unwavering support for one's country
\item \textbf{Constructive Patriotism ($\eta_{\text{pc}}$)}: Critical, improvement-oriented attachment
\item \textbf{Daily Life Secularism ($\eta_{\text{sdl}}$)}: Preference for separation of religion in daily practices
\item \textbf{Faith \& Prayer Secularism ($\eta_{\text{sfp}}$)}: Preference for separation in religious matters
\end{enumerate}

\subsubsection{Structural Models}

Each latent variable has its own structural equation:
\begin{align}
\eta_{\text{pb},n} &= \gamma_{\text{pb,age}} \cdot \text{age}_n^c + \gamma_{\text{pb,inc}} \cdot \text{inc}_n^c + \zeta_{\text{pb},n} \label{eq:hcm_full_pb} \\
\eta_{\text{pc},n} &= \gamma_{\text{pc,edu}} \cdot \text{edu}_n^c + \zeta_{\text{pc},n} \label{eq:hcm_full_pc} \\
\eta_{\text{sdl},n} &= \gamma_{\text{sdl,edu}} \cdot \text{edu}_n^c + \gamma_{\text{sdl,inc}} \cdot \text{inc}_n^c + \zeta_{\text{sdl},n} \label{eq:hcm_full_sdl} \\
\eta_{\text{sfp},n} &= \gamma_{\text{sfp,edu}} \cdot \text{edu}_n^c + \zeta_{\text{sfp},n} \label{eq:hcm_full_sfp}
\end{align}

\subsubsection{Domain Separation Hypothesis}

A key theoretical contribution is the domain separation hypothesis: different psychological constructs affect different behavioral domains:

\begin{itemize}
\item \textbf{Patriotism $\rightarrow$ Fee sensitivity}: Patriotic attitudes affect willingness to pay for government services
\item \textbf{Secularism $\rightarrow$ Duration sensitivity}: Secular attitudes affect time valuation and patience
\end{itemize}

This yields the choice model specification:
\begin{align}
\beta_{\text{fee},n} &= \beta_{\text{fee}} + \lambda_{\text{pb}} \eta_{\text{pb},n} + \lambda_{\text{pc}} \eta_{\text{pc},n} + \lambda_{\text{sdl}} \eta_{\text{sdl},n} + \lambda_{\text{sfp}} \eta_{\text{sfp},n} \\
\beta_{\text{dur},n} &= \beta_{\text{dur}} + \lambda'_{\text{pb}} \eta_{\text{pb},n} + \lambda'_{\text{pc}} \eta_{\text{pc},n} + \lambda'_{\text{sdl}} \eta_{\text{sdl},n} + \lambda'_{\text{sfp}} \eta_{\text{sfp},n}
\end{align}

\subsubsection{Measurement Structure}

Each latent variable is measured by five Likert indicators (20 indicators total), using ordered probit models as in the basic HCM.

%% ============================================================================
%% SUBSECTION: Model 6 - ICLV
%% ============================================================================
\subsection{Model 6: Integrated Choice and Latent Variable}
\label{sec:iclv}

The Integrated Choice and Latent Variable (ICLV) model addresses the attenuation bias problem inherent in two-stage HCM estimation by simultaneously estimating all model components.

\subsubsection{Simultaneous Estimation}

Unlike two-stage HCM, ICLV estimates the structural, measurement, and choice models jointly. The individual contribution to the likelihood is:
\begin{equation}
L_n = \int \underbrace{P(y_n \,|\, \boldsymbol{\eta})}_{\text{Choice likelihood}} \times \underbrace{\prod_{k=1}^{K} P(I_{nk} \,|\, \boldsymbol{\eta})}_{\text{Measurement likelihood}} \times \underbrace{f(\boldsymbol{\eta} \,|\, \mathbf{z}_n)}_{\text{Structural density}} \, d\boldsymbol{\eta}
\label{eq:iclv_likelihood}
\end{equation}

\subsubsection{Components}

\textbf{Choice Likelihood}: Standard logit probability conditional on latent variables:
\begin{equation}
P(y_n = j \,|\, \boldsymbol{\eta}) = \frac{\exp(V_{nj}(\boldsymbol{\eta}))}{\sum_k \exp(V_{nk}(\boldsymbol{\eta}))}
\end{equation}

\textbf{Measurement Likelihood}: Product of ordered probit probabilities across indicators:
\begin{equation}
\prod_{k} P(I_{nk} \,|\, \boldsymbol{\eta}) = \prod_{k} \left[ \Phi(\tau_{k,c} - \lambda_k \eta) - \Phi(\tau_{k,c-1} - \lambda_k \eta) \right]^{\mathbf{1}_{I_{nk}=c}}
\end{equation}

\textbf{Structural Density}: Multivariate normal for latent variables:
\begin{equation}
f(\boldsymbol{\eta} \,|\, \mathbf{z}_n) = \phi(\boldsymbol{\eta} \,|\, \boldsymbol{\Gamma} \mathbf{z}_n, \boldsymbol{\Omega})
\end{equation}

\subsubsection{Monte Carlo Integration}

The integral in Equation~\eqref{eq:iclv_likelihood} is approximated via Monte Carlo simulation:
\begin{equation}
\tilde{L}_n = \frac{1}{R} \sum_{r=1}^{R} P(y_n \,|\, \boldsymbol{\eta}^{(r)}) \times \prod_{k} P(I_{nk} \,|\, \boldsymbol{\eta}^{(r)})
\label{eq:iclv_mc}
\end{equation}
where $\boldsymbol{\eta}^{(r)} \sim f(\boldsymbol{\eta} \,|\, \mathbf{z}_n)$ are draws from the conditional latent variable distribution.

\subsubsection{Bias Elimination}

By integrating over the distribution of latent variables rather than conditioning on point estimates, ICLV properly accounts for measurement uncertainty. Our validation studies demonstrate that ICLV reduces bias from 15--35\% (two-stage) to approximately 2--5\% (simultaneous), representing a substantial improvement in estimation accuracy.

%% ============================================================================
%% SUBSECTION: Model Progression Rationale
%% ============================================================================
\subsection{Model Progression Rationale}
\label{sec:progression}

The six-model progression follows a deliberate logic, with each specification addressing specific limitations of its predecessor.

\subsubsection{Progression Logic}

\begin{enumerate}
\item \textbf{MNL Basic $\rightarrow$ MNL Demographics}: Addresses the unrealistic assumption of homogeneous preferences by allowing observed characteristics to explain taste variation
\item \textbf{MNL Demographics $\rightarrow$ MXL}: Captures preference heterogeneity that \textit{cannot} be explained by observed demographics
\item \textbf{MXL $\rightarrow$ HCM Basic}: Provides a \textit{theoretical explanation} for unobserved heterogeneity through psychological constructs
\item \textbf{HCM Basic $\rightarrow$ HCM Full}: Enriches the psychological framework with multiple constructs and domain-specific effects
\item \textbf{HCM Full $\rightarrow$ ICLV}: Eliminates the attenuation bias inherent in two-stage estimation
\end{enumerate}

\subsubsection{Summary Comparison}

Table~\ref{tab:model_summary} provides a comprehensive comparison of the six specifications.

\begin{table}[htbp]
\centering
\caption{Comprehensive Model Comparison}
\label{tab:model_summary}
\begin{threeparttable}
\begin{tabular}{lcccccc}
\toprule
Feature & MNL & MNL-D & MXL & HCM-B & HCM-F & ICLV \\
\midrule
Observable heterogeneity & & \checkmark & & & & \\
Unobserved heterogeneity & & & \checkmark & \checkmark & \checkmark & \checkmark \\
Latent variables & & & & \checkmark & \checkmark & \checkmark \\
Multiple LVs & & & & & \checkmark & \checkmark \\
Simultaneous estimation & & & & & & \checkmark \\
\addlinespace
Relaxes IIA & & & \checkmark & \checkmark & \checkmark & \checkmark \\
Unbiased LV estimates & N/A & N/A & N/A & & & \checkmark \\
\addlinespace
Parameters (choice) & 3 & 8 & 4 & 4 & 11 & 4--11 \\
Estimation method & MLE & MLE & SML & Two-stage & Two-stage & SML \\
\bottomrule
\end{tabular}
\begin{tablenotes}
\small
\item Note: MNL = MNL Basic; MNL-D = MNL Demographics; HCM-B = HCM Basic; HCM-F = HCM Full. \checkmark\ indicates feature is present. MLE = Maximum Likelihood; SML = Simulated Maximum Likelihood.
\end{tablenotes}
\end{threeparttable}
\end{table}

\subsubsection{Recommended Usage}

The choice among specifications depends on the research objectives:

\begin{itemize}
\item \textbf{Baseline comparison}: MNL Basic provides a benchmark for more complex models
\item \textbf{Policy targeting}: MNL Demographics identifies demographic segments with different preferences
\item \textbf{Heterogeneity quantification}: MXL measures the extent of unexplained preference variation
\item \textbf{Theoretical testing}: HCM tests hypotheses about psychological determinants of choice
\item \textbf{Publication-quality estimates}: ICLV provides unbiased latent variable effects for substantive interpretation
\end{itemize}

%% ============================================================================
%% SUBSECTION: Validation Approach
%% ============================================================================
\subsection{Validation Approach}
\label{sec:validation_approach}

Each model specification is validated using a matching data generating process (DGP) that ensures the simulated data conform exactly to the model's assumptions.

\subsubsection{Parameter Recovery Protocol}

For each model, we:
\begin{enumerate}
\item Specify true parameter values in a configuration file
\item Generate synthetic data using the model's exact DGP
\item Estimate the model on synthetic data
\item Compare estimated to true parameters
\end{enumerate}

\subsubsection{Validation Metrics}

We assess parameter recovery using:
\begin{itemize}
\item \textbf{Percentage bias}: $100 \times (\hat{\theta} - \theta^*) / |\theta^*|$
\item \textbf{95\% CI coverage}: Proportion of estimates where $\theta^* \in [\hat{\theta} \pm 1.96 \times \text{SE}]$
\item \textbf{Significance}: $|t| = |\hat{\theta}/\text{SE}| > 1.96$
\end{itemize}

\subsubsection{Expected Performance}

Well-specified models should achieve:
\begin{itemize}
\item Bias $< 10\%$ for all parameters
\item Coverage $\approx 95\%$ (acceptable range: 90--97\%)
\item All parameters significant at $\alpha = 0.05$
\end{itemize}

The ICLV model, by eliminating attenuation bias, achieves substantially better performance on latent variable coefficients compared to two-stage HCM alternatives.

%% ============================================================================
%% End of Discrete Choice Model Framework Section
%% ============================================================================

%%
%% Required packages in main document:
%%   \usepackage{amsmath}
%%   \usepackage{amssymb}
%%   \usepackage{booktabs}
%%   \usepackage{threeparttable}
%%   \usepackage{multirow}

\section{Discrete Choice Model Framework}
\label{sec:dcm_framework}

This section presents the discrete choice modeling framework employed in our empirical analysis. We implement a sequence of six model specifications, each building upon its predecessor to address increasingly sophisticated forms of preference heterogeneity. The progression from basic multinomial logit through integrated choice and latent variable models provides both methodological rigor and substantive insights into the determinants of choice behavior.

%% ============================================================================
%% SUBSECTION: Framework Overview
%% ============================================================================
\subsection{Framework Overview}
\label{sec:framework_overview}

Our modeling framework comprises six discrete choice specifications organized in a deliberate hierarchy of complexity. Each model addresses a specific limitation of its predecessor while maintaining consistency in core structural assumptions. Table~\ref{tab:model_hierarchy} summarizes this progression.

\begin{table}[htbp]
\centering
\caption{Model Hierarchy and Heterogeneity Treatment}
\label{tab:model_hierarchy}
\begin{threeparttable}
\begin{tabular}{llccl}
\toprule
Model & Type & $K$ & Heterogeneity & Key Innovation \\
\midrule
1. MNL Basic & Baseline & 3 & None & Foundation specification \\
2. MNL Demographics & Observable & 8 & Demographics & Taste-demographic interactions \\
3. MXL Basic & Random Coef. & 4 & Unobserved & Distributional heterogeneity \\
4. HCM Basic & Hybrid & 5+ & Latent (single) & Psychological constructs \\
5. HCM Full & Hybrid & 11+ & Latent (multiple) & Domain-specific effects \\
6. ICLV & Integrated & 10+ & Latent (simult.) & Bias elimination \\
\bottomrule
\end{tabular}
\begin{tablenotes}
\small
\item Note: $K$ = number of estimated parameters (excluding measurement model parameters for HCM/ICLV). MNL = Multinomial Logit; MXL = Mixed Logit; HCM = Hybrid Choice Model; ICLV = Integrated Choice and Latent Variable.
\end{tablenotes}
\end{threeparttable}
\end{table}

The framework is designed around a common choice context: individuals choosing among three service alternatives---two paid expedited options and one standard (free) option with longer processing time. Attributes include processing fee (in Turkish Lira) and processing duration (in weeks). This unified context enables direct comparison of model performance and parameter estimates across specifications.

\subsubsection{Data Structure}

Each model operates on panel data with the following structure:
\begin{itemize}
\item $N = 500$ individuals
\item $T = 10$ choice tasks per individual
\item $J = 3$ alternatives per choice task
\item Total observations: $N \times T = 5{,}000$ choices
\end{itemize}

Attributes are drawn from a stratified-factorial experimental design that ensures orthogonality between fee and duration, enabling precise identification of both effects.

\subsubsection{Common Notation}

Throughout this section, we employ the following notation:
\begin{itemize}
\item $V_{nj}$ = Systematic utility of alternative $j$ for individual $n$
\item $\beta_{\text{fee}}$ = Marginal utility of fee (expected negative)
\item $\beta_{\text{dur}}$ = Marginal utility of duration (expected negative)
\item $\text{ASC}_{\text{paid}}$ = Alternative-specific constant for paid options
\item $\eta$ = Latent variable (psychological construct)
\item $\lambda$ = Factor loading (measurement model)
\item $\tau$ = Threshold parameter (ordered probit)
\end{itemize}

%% ============================================================================
%% SUBSECTION: Model 1 - MNL Basic
%% ============================================================================
\subsection{Model 1: Multinomial Logit Basic}
\label{sec:mnl_basic}

The multinomial logit (MNL) basic model serves as the foundation of our analysis, providing a parsimonious baseline against which more complex specifications are compared.

\subsubsection{Model Specification}

Under the assumption of homogeneous preferences, all individuals share identical taste parameters. The systematic utility for alternative $j$ is:
\begin{equation}
V_{nj} = \text{ASC}_{\text{paid}} \cdot \mathbf{1}_{j \in \{1,2\}} + \beta_{\text{fee}} \cdot \text{fee}_j + \beta_{\text{dur}} \cdot \text{dur}_j
\label{eq:mnl_basic_utility}
\end{equation}
where $\mathbf{1}_{j \in \{1,2\}}$ is an indicator function equal to 1 for paid alternatives and 0 for the standard option (which serves as the reference category with ASC normalized to zero).

\subsubsection{Choice Probability}

Given IID Gumbel-distributed errors, the choice probability takes the standard logit form:
\begin{equation}
P_{nj} = \frac{\exp(V_{nj})}{\exp(V_{n1}) + \exp(V_{n2}) + \exp(V_{n3})}
\label{eq:mnl_basic_prob}
\end{equation}

\subsubsection{Parameters}

The model estimates three parameters:

\begin{table}[htbp]
\centering
\caption{MNL Basic Parameters}
\label{tab:mnl_basic_params}
\begin{tabular}{llc}
\toprule
Parameter & Interpretation & Expected Sign \\
\midrule
$\text{ASC}_{\text{paid}}$ & Baseline preference for expedited service & + \\
$\beta_{\text{fee}}$ & Marginal disutility of fee (per 10,000 TL) & $-$ \\
$\beta_{\text{dur}}$ & Marginal disutility of duration (per week) & $-$ \\
\bottomrule
\end{tabular}
\end{table}

\subsubsection{Limitations}

The MNL basic model assumes that all individuals have identical preferences---a strong assumption that is unlikely to hold empirically. Observable differences in age, income, and education are ignored, as are unobserved psychological factors that may systematically influence choice.

%% ============================================================================
%% SUBSECTION: Model 2 - MNL Demographics
%% ============================================================================
\subsection{Model 2: Multinomial Logit with Demographics}
\label{sec:mnl_demo}

The MNL demographics model relaxes the homogeneity assumption by allowing taste parameters to vary as a function of observed individual characteristics.

\subsubsection{Model Specification}

We specify individual-specific coefficients that depend on demographic variables:
\begin{align}
\beta_{\text{fee},n} &= \beta_{\text{fee}} + \beta_{\text{fee,age}} \cdot \text{age}_n^c + \beta_{\text{fee,edu}} \cdot \text{edu}_n^c + \beta_{\text{fee,inc}} \cdot \text{inc}_n^c \label{eq:fee_demo} \\
\beta_{\text{dur},n} &= \beta_{\text{dur}} + \beta_{\text{dur,edu}} \cdot \text{edu}_n^c + \beta_{\text{dur,inc}} \cdot \text{inc}_n^c \label{eq:dur_demo}
\end{align}
where superscript $c$ denotes centered and scaled demographic variables.

The utility specification becomes:
\begin{equation}
V_{nj} = \text{ASC}_{\text{paid}} \cdot \mathbf{1}_{j \in \{1,2\}} + \beta_{\text{fee},n} \cdot \text{fee}_j + \beta_{\text{dur},n} \cdot \text{dur}_j
\label{eq:mnl_demo_utility}
\end{equation}

\subsubsection{Asymmetric Interaction Structure}

A notable feature of this specification is the asymmetric treatment of demographic interactions:
\begin{itemize}
\item \textbf{Fee sensitivity}: Varies with age, education, \textit{and} income
\item \textbf{Duration sensitivity}: Varies with education and income only (not age)
\end{itemize}

This asymmetry reflects theoretical considerations:
\begin{itemize}
\item \textbf{Age $\rightarrow$ Fee}: Older individuals may have accumulated wealth, reducing fee sensitivity
\item \textbf{Education/Income $\rightarrow$ Fee}: Higher socioeconomic status reduces price sensitivity
\item \textbf{Education/Income $\rightarrow$ Duration}: Higher opportunity cost of time increases duration sensitivity
\item \textbf{Age $\not\rightarrow$ Duration}: Time valuation driven by opportunity cost, not lifecycle stage
\end{itemize}

\subsubsection{Parameters}

The model estimates eight parameters:

\begin{table}[htbp]
\centering
\caption{MNL Demographics Parameters}
\label{tab:mnl_demo_params}
\begin{tabular}{llcc}
\toprule
Parameter & Interpretation & Expected Sign & Applies To \\
\midrule
$\text{ASC}_{\text{paid}}$ & Baseline expedited preference & + & All \\
$\beta_{\text{fee}}$ & Base fee sensitivity & $-$ & All \\
$\beta_{\text{fee,age}}$ & Age effect on fee sensitivity & + & Fee \\
$\beta_{\text{fee,edu}}$ & Education effect on fee sensitivity & + & Fee \\
$\beta_{\text{fee,inc}}$ & Income effect on fee sensitivity & + & Fee \\
$\beta_{\text{dur}}$ & Base duration sensitivity & $-$ & All \\
$\beta_{\text{dur,edu}}$ & Education effect on duration sensitivity & $-$ & Duration \\
$\beta_{\text{dur,inc}}$ & Income effect on duration sensitivity & $-$ & Duration \\
\bottomrule
\end{tabular}
\end{table}

Positive interaction coefficients on fee indicate \textit{reduced} fee sensitivity (less negative effective $\beta_{\text{fee}}$). Negative interaction coefficients on duration indicate \textit{increased} duration sensitivity (more negative effective $\beta_{\text{dur}}$).

%% ============================================================================
%% SUBSECTION: Model 3 - Mixed Logit
%% ============================================================================
\subsection{Model 3: Mixed Logit Basic}
\label{sec:mxl_basic}

The mixed logit (MXL) model captures unobserved preference heterogeneity by allowing taste parameters to follow continuous distributions across the population.

\subsubsection{Model Specification}

We specify the fee coefficient as normally distributed:
\begin{equation}
\beta_{\text{fee},n} \sim N(\mu_{\text{fee}}, \sigma^2_{\text{fee}})
\label{eq:mxl_distribution}
\end{equation}

Equivalently, using the standard normal draw representation:
\begin{equation}
\beta_{\text{fee},n} = \mu_{\text{fee}} + \sigma_{\text{fee}} \cdot \nu_n, \quad \nu_n \sim N(0,1)
\label{eq:mxl_draw}
\end{equation}

The utility specification is:
\begin{equation}
V_{nj} = \text{ASC}_{\text{paid}} \cdot \mathbf{1}_{j \in \{1,2\}} + \beta_{\text{fee},n} \cdot \text{fee}_j + \beta_{\text{dur}} \cdot \text{dur}_j
\label{eq:mxl_utility}
\end{equation}

Note that $\beta_{\text{dur}}$ is specified as fixed (non-random) to ensure identification and parsimony.

\subsubsection{Choice Probability}

The unconditional choice probability requires integration over the mixing distribution:
\begin{equation}
P_{nj} = \int \frac{\exp(V_{nj}(\beta))}{\sum_k \exp(V_{nk}(\beta))} \cdot \phi\left(\frac{\beta - \mu}{\sigma}\right) \cdot \frac{1}{\sigma} \, d\beta
\label{eq:mxl_prob}
\end{equation}

This integral lacks a closed-form solution. We approximate it using simulated maximum likelihood with $R = 500$ Halton draws:
\begin{equation}
\tilde{P}_{nj} = \frac{1}{R} \sum_{r=1}^{R} \frac{\exp(V_{nj}(\beta^{(r)}_n))}{\sum_k \exp(V_{nk}(\beta^{(r)}_n))}
\label{eq:mxl_simulated}
\end{equation}

\subsubsection{Parameters}

The model estimates four parameters:

\begin{table}[htbp]
\centering
\caption{MXL Basic Parameters}
\label{tab:mxl_params}
\begin{tabular}{llc}
\toprule
Parameter & Interpretation & Expected Sign \\
\midrule
$\text{ASC}_{\text{paid}}$ & Baseline expedited preference & + \\
$\mu_{\text{fee}}$ & Mean fee coefficient & $-$ \\
$\sigma_{\text{fee}}$ & Standard deviation of fee coefficient & + \\
$\beta_{\text{dur}}$ & Duration coefficient (fixed) & $-$ \\
\bottomrule
\end{tabular}
\end{table}

\subsubsection{Interpretation of Heterogeneity}

The coefficient of variation $\text{CV} = \sigma_{\text{fee}}/|\mu_{\text{fee}}|$ provides a normalized measure of heterogeneity. Under the normal specification:
\begin{itemize}
\item If $\text{CV} = 0.4$, approximately 1\% of the population has positive $\beta_{\text{fee}}$ (counterintuitive)
\item Large $\sigma$ relative to $|\mu|$ may indicate the need for alternative distributions (e.g., lognormal, constrained normal)
\end{itemize}

%% ============================================================================
%% SUBSECTION: Model 4 - HCM Basic
%% ============================================================================
\subsection{Model 4: Hybrid Choice Model Basic}
\label{sec:hcm_basic}

The hybrid choice model (HCM) introduces latent psychological constructs as determinants of choice behavior. The basic HCM specification incorporates a single latent variable.

\subsubsection{Conceptual Framework}

HCM extends the standard choice model by positing that unobserved psychological attitudes---measured imperfectly through survey indicators---systematically influence taste parameters. The model comprises three interconnected components:

\begin{enumerate}
\item \textbf{Structural model}: Links demographics to latent variables
\item \textbf{Measurement model}: Relates latent variables to observed indicators
\item \textbf{Choice model}: Incorporates latent variables as taste shifters
\end{enumerate}

\subsubsection{Structural Model}

The latent variable (Blind Patriotism, $\eta_{\text{pb}}$) is determined by demographics:
\begin{equation}
\eta_{\text{pb},n} = \gamma_0 + \gamma_{\text{age}} \cdot (\text{age}_n - \bar{\text{age}}) + \zeta_n, \quad \zeta_n \sim N(0, \sigma^2_\zeta)
\label{eq:hcm_structural}
\end{equation}

This specification reflects the hypothesis that patriotic attitudes vary systematically with age, with a residual component capturing individual-specific variation.

\subsubsection{Measurement Model}

The latent variable is measured through five Likert-scale indicators ($I_1, \ldots, I_5$), each following an ordered probit specification:
\begin{equation}
P(I_{nk} = c \,|\, \eta_n) = \Phi(\tau_{k,c} - \lambda_k \eta_n) - \Phi(\tau_{k,c-1} - \lambda_k \eta_n)
\label{eq:hcm_measurement}
\end{equation}
where:
\begin{itemize}
\item $\lambda_k$ = factor loading for indicator $k$ (first loading fixed to 1 for identification)
\item $\tau_{k,c}$ = threshold for category $c$ of indicator $k$
\item $\Phi(\cdot)$ = standard normal CDF
\end{itemize}

\subsubsection{Choice Model}

The latent variable enters the choice model as a taste shifter:
\begin{align}
\beta_{\text{fee},n} &= \beta_{\text{fee}} + \lambda_{\text{fee,pb}} \cdot \eta_{\text{pb},n} \label{eq:hcm_fee} \\
V_{nj} &= \text{ASC}_{\text{paid}} \cdot \mathbf{1}_{j \in \{1,2\}} + \beta_{\text{fee},n} \cdot \text{fee}_j + \beta_{\text{dur}} \cdot \text{dur}_j \label{eq:hcm_utility}
\end{align}

Negative $\lambda_{\text{fee,pb}}$ indicates that higher Blind Patriotism reduces fee sensitivity (individuals with stronger patriotic attitudes are more willing to pay for government services).

\subsubsection{Estimation Challenge: Attenuation Bias}

When estimated via two-stage procedures (factor analysis followed by choice model estimation), HCM suffers from attenuation bias. The estimated latent variable effect is systematically underestimated by 15--30\% due to measurement error in the first-stage latent scores.

%% ============================================================================
%% SUBSECTION: Model 5 - HCM Full
%% ============================================================================
\subsection{Model 5: Hybrid Choice Model Full}
\label{sec:hcm_full}

The full HCM specification extends the basic model to incorporate four latent psychological constructs with domain-specific effects.

\subsubsection{Four Latent Variables}

We model four attitudinal constructs:

\begin{enumerate}
\item \textbf{Blind Patriotism ($\eta_{\text{pb}}$)}: Uncritical, unwavering support for one's country
\item \textbf{Constructive Patriotism ($\eta_{\text{pc}}$)}: Critical, improvement-oriented attachment
\item \textbf{Daily Life Secularism ($\eta_{\text{sdl}}$)}: Preference for separation of religion in daily practices
\item \textbf{Faith \& Prayer Secularism ($\eta_{\text{sfp}}$)}: Preference for separation in religious matters
\end{enumerate}

\subsubsection{Structural Models}

Each latent variable has its own structural equation:
\begin{align}
\eta_{\text{pb},n} &= \gamma_{\text{pb,age}} \cdot \text{age}_n^c + \gamma_{\text{pb,inc}} \cdot \text{inc}_n^c + \zeta_{\text{pb},n} \label{eq:hcm_full_pb} \\
\eta_{\text{pc},n} &= \gamma_{\text{pc,edu}} \cdot \text{edu}_n^c + \zeta_{\text{pc},n} \label{eq:hcm_full_pc} \\
\eta_{\text{sdl},n} &= \gamma_{\text{sdl,edu}} \cdot \text{edu}_n^c + \gamma_{\text{sdl,inc}} \cdot \text{inc}_n^c + \zeta_{\text{sdl},n} \label{eq:hcm_full_sdl} \\
\eta_{\text{sfp},n} &= \gamma_{\text{sfp,edu}} \cdot \text{edu}_n^c + \zeta_{\text{sfp},n} \label{eq:hcm_full_sfp}
\end{align}

\subsubsection{Domain Separation Hypothesis}

A key theoretical contribution is the domain separation hypothesis: different psychological constructs affect different behavioral domains:

\begin{itemize}
\item \textbf{Patriotism $\rightarrow$ Fee sensitivity}: Patriotic attitudes affect willingness to pay for government services
\item \textbf{Secularism $\rightarrow$ Duration sensitivity}: Secular attitudes affect time valuation and patience
\end{itemize}

This yields the choice model specification:
\begin{align}
\beta_{\text{fee},n} &= \beta_{\text{fee}} + \lambda_{\text{pb}} \eta_{\text{pb},n} + \lambda_{\text{pc}} \eta_{\text{pc},n} + \lambda_{\text{sdl}} \eta_{\text{sdl},n} + \lambda_{\text{sfp}} \eta_{\text{sfp},n} \\
\beta_{\text{dur},n} &= \beta_{\text{dur}} + \lambda'_{\text{pb}} \eta_{\text{pb},n} + \lambda'_{\text{pc}} \eta_{\text{pc},n} + \lambda'_{\text{sdl}} \eta_{\text{sdl},n} + \lambda'_{\text{sfp}} \eta_{\text{sfp},n}
\end{align}

\subsubsection{Measurement Structure}

Each latent variable is measured by five Likert indicators (20 indicators total), using ordered probit models as in the basic HCM.

%% ============================================================================
%% SUBSECTION: Model 6 - ICLV
%% ============================================================================
\subsection{Model 6: Integrated Choice and Latent Variable}
\label{sec:iclv}

The Integrated Choice and Latent Variable (ICLV) model addresses the attenuation bias problem inherent in two-stage HCM estimation by simultaneously estimating all model components.

\subsubsection{Simultaneous Estimation}

Unlike two-stage HCM, ICLV estimates the structural, measurement, and choice models jointly. The individual contribution to the likelihood is:
\begin{equation}
L_n = \int \underbrace{P(y_n \,|\, \boldsymbol{\eta})}_{\text{Choice likelihood}} \times \underbrace{\prod_{k=1}^{K} P(I_{nk} \,|\, \boldsymbol{\eta})}_{\text{Measurement likelihood}} \times \underbrace{f(\boldsymbol{\eta} \,|\, \mathbf{z}_n)}_{\text{Structural density}} \, d\boldsymbol{\eta}
\label{eq:iclv_likelihood}
\end{equation}

\subsubsection{Components}

\textbf{Choice Likelihood}: Standard logit probability conditional on latent variables:
\begin{equation}
P(y_n = j \,|\, \boldsymbol{\eta}) = \frac{\exp(V_{nj}(\boldsymbol{\eta}))}{\sum_k \exp(V_{nk}(\boldsymbol{\eta}))}
\end{equation}

\textbf{Measurement Likelihood}: Product of ordered probit probabilities across indicators:
\begin{equation}
\prod_{k} P(I_{nk} \,|\, \boldsymbol{\eta}) = \prod_{k} \left[ \Phi(\tau_{k,c} - \lambda_k \eta) - \Phi(\tau_{k,c-1} - \lambda_k \eta) \right]^{\mathbf{1}_{I_{nk}=c}}
\end{equation}

\textbf{Structural Density}: Multivariate normal for latent variables:
\begin{equation}
f(\boldsymbol{\eta} \,|\, \mathbf{z}_n) = \phi(\boldsymbol{\eta} \,|\, \boldsymbol{\Gamma} \mathbf{z}_n, \boldsymbol{\Omega})
\end{equation}

\subsubsection{Monte Carlo Integration}

The integral in Equation~\eqref{eq:iclv_likelihood} is approximated via Monte Carlo simulation:
\begin{equation}
\tilde{L}_n = \frac{1}{R} \sum_{r=1}^{R} P(y_n \,|\, \boldsymbol{\eta}^{(r)}) \times \prod_{k} P(I_{nk} \,|\, \boldsymbol{\eta}^{(r)})
\label{eq:iclv_mc}
\end{equation}
where $\boldsymbol{\eta}^{(r)} \sim f(\boldsymbol{\eta} \,|\, \mathbf{z}_n)$ are draws from the conditional latent variable distribution.

\subsubsection{Bias Elimination}

By integrating over the distribution of latent variables rather than conditioning on point estimates, ICLV properly accounts for measurement uncertainty. Our validation studies demonstrate that ICLV reduces bias from 15--35\% (two-stage) to approximately 2--5\% (simultaneous), representing a substantial improvement in estimation accuracy.

%% ============================================================================
%% SUBSECTION: Model Progression Rationale
%% ============================================================================
\subsection{Model Progression Rationale}
\label{sec:progression}

The six-model progression follows a deliberate logic, with each specification addressing specific limitations of its predecessor.

\subsubsection{Progression Logic}

\begin{enumerate}
\item \textbf{MNL Basic $\rightarrow$ MNL Demographics}: Addresses the unrealistic assumption of homogeneous preferences by allowing observed characteristics to explain taste variation
\item \textbf{MNL Demographics $\rightarrow$ MXL}: Captures preference heterogeneity that \textit{cannot} be explained by observed demographics
\item \textbf{MXL $\rightarrow$ HCM Basic}: Provides a \textit{theoretical explanation} for unobserved heterogeneity through psychological constructs
\item \textbf{HCM Basic $\rightarrow$ HCM Full}: Enriches the psychological framework with multiple constructs and domain-specific effects
\item \textbf{HCM Full $\rightarrow$ ICLV}: Eliminates the attenuation bias inherent in two-stage estimation
\end{enumerate}

\subsubsection{Summary Comparison}

Table~\ref{tab:model_summary} provides a comprehensive comparison of the six specifications.

\begin{table}[htbp]
\centering
\caption{Comprehensive Model Comparison}
\label{tab:model_summary}
\begin{threeparttable}
\begin{tabular}{lcccccc}
\toprule
Feature & MNL & MNL-D & MXL & HCM-B & HCM-F & ICLV \\
\midrule
Observable heterogeneity & & \checkmark & & & & \\
Unobserved heterogeneity & & & \checkmark & \checkmark & \checkmark & \checkmark \\
Latent variables & & & & \checkmark & \checkmark & \checkmark \\
Multiple LVs & & & & & \checkmark & \checkmark \\
Simultaneous estimation & & & & & & \checkmark \\
\addlinespace
Relaxes IIA & & & \checkmark & \checkmark & \checkmark & \checkmark \\
Unbiased LV estimates & N/A & N/A & N/A & & & \checkmark \\
\addlinespace
Parameters (choice) & 3 & 8 & 4 & 4 & 11 & 4--11 \\
Estimation method & MLE & MLE & SML & Two-stage & Two-stage & SML \\
\bottomrule
\end{tabular}
\begin{tablenotes}
\small
\item Note: MNL = MNL Basic; MNL-D = MNL Demographics; HCM-B = HCM Basic; HCM-F = HCM Full. \checkmark\ indicates feature is present. MLE = Maximum Likelihood; SML = Simulated Maximum Likelihood.
\end{tablenotes}
\end{threeparttable}
\end{table}

\subsubsection{Recommended Usage}

The choice among specifications depends on the research objectives:

\begin{itemize}
\item \textbf{Baseline comparison}: MNL Basic provides a benchmark for more complex models
\item \textbf{Policy targeting}: MNL Demographics identifies demographic segments with different preferences
\item \textbf{Heterogeneity quantification}: MXL measures the extent of unexplained preference variation
\item \textbf{Theoretical testing}: HCM tests hypotheses about psychological determinants of choice
\item \textbf{Publication-quality estimates}: ICLV provides unbiased latent variable effects for substantive interpretation
\end{itemize}

%% ============================================================================
%% SUBSECTION: Validation Approach
%% ============================================================================
\subsection{Validation Approach}
\label{sec:validation_approach}

Each model specification is validated using a matching data generating process (DGP) that ensures the simulated data conform exactly to the model's assumptions.

\subsubsection{Parameter Recovery Protocol}

For each model, we:
\begin{enumerate}
\item Specify true parameter values in a configuration file
\item Generate synthetic data using the model's exact DGP
\item Estimate the model on synthetic data
\item Compare estimated to true parameters
\end{enumerate}

\subsubsection{Validation Metrics}

We assess parameter recovery using:
\begin{itemize}
\item \textbf{Percentage bias}: $100 \times (\hat{\theta} - \theta^*) / |\theta^*|$
\item \textbf{95\% CI coverage}: Proportion of estimates where $\theta^* \in [\hat{\theta} \pm 1.96 \times \text{SE}]$
\item \textbf{Significance}: $|t| = |\hat{\theta}/\text{SE}| > 1.96$
\end{itemize}

\subsubsection{Expected Performance}

Well-specified models should achieve:
\begin{itemize}
\item Bias $< 10\%$ for all parameters
\item Coverage $\approx 95\%$ (acceptable range: 90--97\%)
\item All parameters significant at $\alpha = 0.05$
\end{itemize}

The ICLV model, by eliminating attenuation bias, achieves substantially better performance on latent variable coefficients compared to two-stage HCM alternatives.

%% ============================================================================
%% End of Discrete Choice Model Framework Section
%% ============================================================================

%%
%% Required packages in main document:
%%   \usepackage{amsmath}
%%   \usepackage{amssymb}
%%   \usepackage{booktabs}
%%   \usepackage{threeparttable}
%%   \usepackage{multirow}

\section{Discrete Choice Model Framework}
\label{sec:dcm_framework}

This section presents the discrete choice modeling framework employed in our empirical analysis. We implement a sequence of six model specifications, each building upon its predecessor to address increasingly sophisticated forms of preference heterogeneity. The progression from basic multinomial logit through integrated choice and latent variable models provides both methodological rigor and substantive insights into the determinants of choice behavior.

%% ============================================================================
%% SUBSECTION: Framework Overview
%% ============================================================================
\subsection{Framework Overview}
\label{sec:framework_overview}

Our modeling framework comprises six discrete choice specifications organized in a deliberate hierarchy of complexity. Each model addresses a specific limitation of its predecessor while maintaining consistency in core structural assumptions. Table~\ref{tab:model_hierarchy} summarizes this progression.

\begin{table}[htbp]
\centering
\caption{Model Hierarchy and Heterogeneity Treatment}
\label{tab:model_hierarchy}
\begin{threeparttable}
\begin{tabular}{llccl}
\toprule
Model & Type & $K$ & Heterogeneity & Key Innovation \\
\midrule
1. MNL Basic & Baseline & 3 & None & Foundation specification \\
2. MNL Demographics & Observable & 8 & Demographics & Taste-demographic interactions \\
3. MXL Basic & Random Coef. & 4 & Unobserved & Distributional heterogeneity \\
4. HCM Basic & Hybrid & 5+ & Latent (single) & Psychological constructs \\
5. HCM Full & Hybrid & 11+ & Latent (multiple) & Domain-specific effects \\
6. ICLV & Integrated & 10+ & Latent (simult.) & Bias elimination \\
\bottomrule
\end{tabular}
\begin{tablenotes}
\small
\item Note: $K$ = number of estimated parameters (excluding measurement model parameters for HCM/ICLV). MNL = Multinomial Logit; MXL = Mixed Logit; HCM = Hybrid Choice Model; ICLV = Integrated Choice and Latent Variable.
\end{tablenotes}
\end{threeparttable}
\end{table}

The framework is designed around a common choice context: individuals choosing among three service alternatives---two paid expedited options and one standard (free) option with longer processing time. Attributes include processing fee (in Turkish Lira) and processing duration (in weeks). This unified context enables direct comparison of model performance and parameter estimates across specifications.

\subsubsection{Data Structure}

Each model operates on panel data with the following structure:
\begin{itemize}
\item $N = 500$ individuals
\item $T = 10$ choice tasks per individual
\item $J = 3$ alternatives per choice task
\item Total observations: $N \times T = 5{,}000$ choices
\end{itemize}

Attributes are drawn from a stratified-factorial experimental design that ensures orthogonality between fee and duration, enabling precise identification of both effects.

\subsubsection{Common Notation}

Throughout this section, we employ the following notation:
\begin{itemize}
\item $V_{nj}$ = Systematic utility of alternative $j$ for individual $n$
\item $\beta_{\text{fee}}$ = Marginal utility of fee (expected negative)
\item $\beta_{\text{dur}}$ = Marginal utility of duration (expected negative)
\item $\text{ASC}_{\text{paid}}$ = Alternative-specific constant for paid options
\item $\eta$ = Latent variable (psychological construct)
\item $\lambda$ = Factor loading (measurement model)
\item $\tau$ = Threshold parameter (ordered probit)
\end{itemize}

%% ============================================================================
%% SUBSECTION: Model 1 - MNL Basic
%% ============================================================================
\subsection{Model 1: Multinomial Logit Basic}
\label{sec:mnl_basic}

The multinomial logit (MNL) basic model serves as the foundation of our analysis, providing a parsimonious baseline against which more complex specifications are compared.

\subsubsection{Model Specification}

Under the assumption of homogeneous preferences, all individuals share identical taste parameters. The systematic utility for alternative $j$ is:
\begin{equation}
V_{nj} = \text{ASC}_{\text{paid}} \cdot \mathbf{1}_{j \in \{1,2\}} + \beta_{\text{fee}} \cdot \text{fee}_j + \beta_{\text{dur}} \cdot \text{dur}_j
\label{eq:mnl_basic_utility}
\end{equation}
where $\mathbf{1}_{j \in \{1,2\}}$ is an indicator function equal to 1 for paid alternatives and 0 for the standard option (which serves as the reference category with ASC normalized to zero).

\subsubsection{Choice Probability}

Given IID Gumbel-distributed errors, the choice probability takes the standard logit form:
\begin{equation}
P_{nj} = \frac{\exp(V_{nj})}{\exp(V_{n1}) + \exp(V_{n2}) + \exp(V_{n3})}
\label{eq:mnl_basic_prob}
\end{equation}

\subsubsection{Parameters}

The model estimates three parameters:

\begin{table}[htbp]
\centering
\caption{MNL Basic Parameters}
\label{tab:mnl_basic_params}
\begin{tabular}{llc}
\toprule
Parameter & Interpretation & Expected Sign \\
\midrule
$\text{ASC}_{\text{paid}}$ & Baseline preference for expedited service & + \\
$\beta_{\text{fee}}$ & Marginal disutility of fee (per 10,000 TL) & $-$ \\
$\beta_{\text{dur}}$ & Marginal disutility of duration (per week) & $-$ \\
\bottomrule
\end{tabular}
\end{table}

\subsubsection{Limitations}

The MNL basic model assumes that all individuals have identical preferences---a strong assumption that is unlikely to hold empirically. Observable differences in age, income, and education are ignored, as are unobserved psychological factors that may systematically influence choice.

%% ============================================================================
%% SUBSECTION: Model 2 - MNL Demographics
%% ============================================================================
\subsection{Model 2: Multinomial Logit with Demographics}
\label{sec:mnl_demo}

The MNL demographics model relaxes the homogeneity assumption by allowing taste parameters to vary as a function of observed individual characteristics.

\subsubsection{Model Specification}

We specify individual-specific coefficients that depend on demographic variables:
\begin{align}
\beta_{\text{fee},n} &= \beta_{\text{fee}} + \beta_{\text{fee,age}} \cdot \text{age}_n^c + \beta_{\text{fee,edu}} \cdot \text{edu}_n^c + \beta_{\text{fee,inc}} \cdot \text{inc}_n^c \label{eq:fee_demo} \\
\beta_{\text{dur},n} &= \beta_{\text{dur}} + \beta_{\text{dur,edu}} \cdot \text{edu}_n^c + \beta_{\text{dur,inc}} \cdot \text{inc}_n^c \label{eq:dur_demo}
\end{align}
where superscript $c$ denotes centered and scaled demographic variables.

The utility specification becomes:
\begin{equation}
V_{nj} = \text{ASC}_{\text{paid}} \cdot \mathbf{1}_{j \in \{1,2\}} + \beta_{\text{fee},n} \cdot \text{fee}_j + \beta_{\text{dur},n} \cdot \text{dur}_j
\label{eq:mnl_demo_utility}
\end{equation}

\subsubsection{Asymmetric Interaction Structure}

A notable feature of this specification is the asymmetric treatment of demographic interactions:
\begin{itemize}
\item \textbf{Fee sensitivity}: Varies with age, education, \textit{and} income
\item \textbf{Duration sensitivity}: Varies with education and income only (not age)
\end{itemize}

This asymmetry reflects theoretical considerations:
\begin{itemize}
\item \textbf{Age $\rightarrow$ Fee}: Older individuals may have accumulated wealth, reducing fee sensitivity
\item \textbf{Education/Income $\rightarrow$ Fee}: Higher socioeconomic status reduces price sensitivity
\item \textbf{Education/Income $\rightarrow$ Duration}: Higher opportunity cost of time increases duration sensitivity
\item \textbf{Age $\not\rightarrow$ Duration}: Time valuation driven by opportunity cost, not lifecycle stage
\end{itemize}

\subsubsection{Parameters}

The model estimates eight parameters:

\begin{table}[htbp]
\centering
\caption{MNL Demographics Parameters}
\label{tab:mnl_demo_params}
\begin{tabular}{llcc}
\toprule
Parameter & Interpretation & Expected Sign & Applies To \\
\midrule
$\text{ASC}_{\text{paid}}$ & Baseline expedited preference & + & All \\
$\beta_{\text{fee}}$ & Base fee sensitivity & $-$ & All \\
$\beta_{\text{fee,age}}$ & Age effect on fee sensitivity & + & Fee \\
$\beta_{\text{fee,edu}}$ & Education effect on fee sensitivity & + & Fee \\
$\beta_{\text{fee,inc}}$ & Income effect on fee sensitivity & + & Fee \\
$\beta_{\text{dur}}$ & Base duration sensitivity & $-$ & All \\
$\beta_{\text{dur,edu}}$ & Education effect on duration sensitivity & $-$ & Duration \\
$\beta_{\text{dur,inc}}$ & Income effect on duration sensitivity & $-$ & Duration \\
\bottomrule
\end{tabular}
\end{table}

Positive interaction coefficients on fee indicate \textit{reduced} fee sensitivity (less negative effective $\beta_{\text{fee}}$). Negative interaction coefficients on duration indicate \textit{increased} duration sensitivity (more negative effective $\beta_{\text{dur}}$).

%% ============================================================================
%% SUBSECTION: Model 3 - Mixed Logit
%% ============================================================================
\subsection{Model 3: Mixed Logit Basic}
\label{sec:mxl_basic}

The mixed logit (MXL) model captures unobserved preference heterogeneity by allowing taste parameters to follow continuous distributions across the population.

\subsubsection{Model Specification}

We specify the fee coefficient as normally distributed:
\begin{equation}
\beta_{\text{fee},n} \sim N(\mu_{\text{fee}}, \sigma^2_{\text{fee}})
\label{eq:mxl_distribution}
\end{equation}

Equivalently, using the standard normal draw representation:
\begin{equation}
\beta_{\text{fee},n} = \mu_{\text{fee}} + \sigma_{\text{fee}} \cdot \nu_n, \quad \nu_n \sim N(0,1)
\label{eq:mxl_draw}
\end{equation}

The utility specification is:
\begin{equation}
V_{nj} = \text{ASC}_{\text{paid}} \cdot \mathbf{1}_{j \in \{1,2\}} + \beta_{\text{fee},n} \cdot \text{fee}_j + \beta_{\text{dur}} \cdot \text{dur}_j
\label{eq:mxl_utility}
\end{equation}

Note that $\beta_{\text{dur}}$ is specified as fixed (non-random) to ensure identification and parsimony.

\subsubsection{Choice Probability}

The unconditional choice probability requires integration over the mixing distribution:
\begin{equation}
P_{nj} = \int \frac{\exp(V_{nj}(\beta))}{\sum_k \exp(V_{nk}(\beta))} \cdot \phi\left(\frac{\beta - \mu}{\sigma}\right) \cdot \frac{1}{\sigma} \, d\beta
\label{eq:mxl_prob}
\end{equation}

This integral lacks a closed-form solution. We approximate it using simulated maximum likelihood with $R = 500$ Halton draws:
\begin{equation}
\tilde{P}_{nj} = \frac{1}{R} \sum_{r=1}^{R} \frac{\exp(V_{nj}(\beta^{(r)}_n))}{\sum_k \exp(V_{nk}(\beta^{(r)}_n))}
\label{eq:mxl_simulated}
\end{equation}

\subsubsection{Parameters}

The model estimates four parameters:

\begin{table}[htbp]
\centering
\caption{MXL Basic Parameters}
\label{tab:mxl_params}
\begin{tabular}{llc}
\toprule
Parameter & Interpretation & Expected Sign \\
\midrule
$\text{ASC}_{\text{paid}}$ & Baseline expedited preference & + \\
$\mu_{\text{fee}}$ & Mean fee coefficient & $-$ \\
$\sigma_{\text{fee}}$ & Standard deviation of fee coefficient & + \\
$\beta_{\text{dur}}$ & Duration coefficient (fixed) & $-$ \\
\bottomrule
\end{tabular}
\end{table}

\subsubsection{Interpretation of Heterogeneity}

The coefficient of variation $\text{CV} = \sigma_{\text{fee}}/|\mu_{\text{fee}}|$ provides a normalized measure of heterogeneity. Under the normal specification:
\begin{itemize}
\item If $\text{CV} = 0.4$, approximately 1\% of the population has positive $\beta_{\text{fee}}$ (counterintuitive)
\item Large $\sigma$ relative to $|\mu|$ may indicate the need for alternative distributions (e.g., lognormal, constrained normal)
\end{itemize}

%% ============================================================================
%% SUBSECTION: Model 4 - HCM Basic
%% ============================================================================
\subsection{Model 4: Hybrid Choice Model Basic}
\label{sec:hcm_basic}

The hybrid choice model (HCM) introduces latent psychological constructs as determinants of choice behavior. The basic HCM specification incorporates a single latent variable.

\subsubsection{Conceptual Framework}

HCM extends the standard choice model by positing that unobserved psychological attitudes---measured imperfectly through survey indicators---systematically influence taste parameters. The model comprises three interconnected components:

\begin{enumerate}
\item \textbf{Structural model}: Links demographics to latent variables
\item \textbf{Measurement model}: Relates latent variables to observed indicators
\item \textbf{Choice model}: Incorporates latent variables as taste shifters
\end{enumerate}

\subsubsection{Structural Model}

The latent variable (Blind Patriotism, $\eta_{\text{pb}}$) is determined by demographics:
\begin{equation}
\eta_{\text{pb},n} = \gamma_0 + \gamma_{\text{age}} \cdot (\text{age}_n - \bar{\text{age}}) + \zeta_n, \quad \zeta_n \sim N(0, \sigma^2_\zeta)
\label{eq:hcm_structural}
\end{equation}

This specification reflects the hypothesis that patriotic attitudes vary systematically with age, with a residual component capturing individual-specific variation.

\subsubsection{Measurement Model}

The latent variable is measured through five Likert-scale indicators ($I_1, \ldots, I_5$), each following an ordered probit specification:
\begin{equation}
P(I_{nk} = c \,|\, \eta_n) = \Phi(\tau_{k,c} - \lambda_k \eta_n) - \Phi(\tau_{k,c-1} - \lambda_k \eta_n)
\label{eq:hcm_measurement}
\end{equation}
where:
\begin{itemize}
\item $\lambda_k$ = factor loading for indicator $k$ (first loading fixed to 1 for identification)
\item $\tau_{k,c}$ = threshold for category $c$ of indicator $k$
\item $\Phi(\cdot)$ = standard normal CDF
\end{itemize}

\subsubsection{Choice Model}

The latent variable enters the choice model as a taste shifter:
\begin{align}
\beta_{\text{fee},n} &= \beta_{\text{fee}} + \lambda_{\text{fee,pb}} \cdot \eta_{\text{pb},n} \label{eq:hcm_fee} \\
V_{nj} &= \text{ASC}_{\text{paid}} \cdot \mathbf{1}_{j \in \{1,2\}} + \beta_{\text{fee},n} \cdot \text{fee}_j + \beta_{\text{dur}} \cdot \text{dur}_j \label{eq:hcm_utility}
\end{align}

Negative $\lambda_{\text{fee,pb}}$ indicates that higher Blind Patriotism reduces fee sensitivity (individuals with stronger patriotic attitudes are more willing to pay for government services).

\subsubsection{Estimation Challenge: Attenuation Bias}

When estimated via two-stage procedures (factor analysis followed by choice model estimation), HCM suffers from attenuation bias. The estimated latent variable effect is systematically underestimated by 15--30\% due to measurement error in the first-stage latent scores.

%% ============================================================================
%% SUBSECTION: Model 5 - HCM Full
%% ============================================================================
\subsection{Model 5: Hybrid Choice Model Full}
\label{sec:hcm_full}

The full HCM specification extends the basic model to incorporate four latent psychological constructs with domain-specific effects.

\subsubsection{Four Latent Variables}

We model four attitudinal constructs:

\begin{enumerate}
\item \textbf{Blind Patriotism ($\eta_{\text{pb}}$)}: Uncritical, unwavering support for one's country
\item \textbf{Constructive Patriotism ($\eta_{\text{pc}}$)}: Critical, improvement-oriented attachment
\item \textbf{Daily Life Secularism ($\eta_{\text{sdl}}$)}: Preference for separation of religion in daily practices
\item \textbf{Faith \& Prayer Secularism ($\eta_{\text{sfp}}$)}: Preference for separation in religious matters
\end{enumerate}

\subsubsection{Structural Models}

Each latent variable has its own structural equation:
\begin{align}
\eta_{\text{pb},n} &= \gamma_{\text{pb,age}} \cdot \text{age}_n^c + \gamma_{\text{pb,inc}} \cdot \text{inc}_n^c + \zeta_{\text{pb},n} \label{eq:hcm_full_pb} \\
\eta_{\text{pc},n} &= \gamma_{\text{pc,edu}} \cdot \text{edu}_n^c + \zeta_{\text{pc},n} \label{eq:hcm_full_pc} \\
\eta_{\text{sdl},n} &= \gamma_{\text{sdl,edu}} \cdot \text{edu}_n^c + \gamma_{\text{sdl,inc}} \cdot \text{inc}_n^c + \zeta_{\text{sdl},n} \label{eq:hcm_full_sdl} \\
\eta_{\text{sfp},n} &= \gamma_{\text{sfp,edu}} \cdot \text{edu}_n^c + \zeta_{\text{sfp},n} \label{eq:hcm_full_sfp}
\end{align}

\subsubsection{Domain Separation Hypothesis}

A key theoretical contribution is the domain separation hypothesis: different psychological constructs affect different behavioral domains:

\begin{itemize}
\item \textbf{Patriotism $\rightarrow$ Fee sensitivity}: Patriotic attitudes affect willingness to pay for government services
\item \textbf{Secularism $\rightarrow$ Duration sensitivity}: Secular attitudes affect time valuation and patience
\end{itemize}

This yields the choice model specification:
\begin{align}
\beta_{\text{fee},n} &= \beta_{\text{fee}} + \lambda_{\text{pb}} \eta_{\text{pb},n} + \lambda_{\text{pc}} \eta_{\text{pc},n} + \lambda_{\text{sdl}} \eta_{\text{sdl},n} + \lambda_{\text{sfp}} \eta_{\text{sfp},n} \\
\beta_{\text{dur},n} &= \beta_{\text{dur}} + \lambda'_{\text{pb}} \eta_{\text{pb},n} + \lambda'_{\text{pc}} \eta_{\text{pc},n} + \lambda'_{\text{sdl}} \eta_{\text{sdl},n} + \lambda'_{\text{sfp}} \eta_{\text{sfp},n}
\end{align}

\subsubsection{Measurement Structure}

Each latent variable is measured by five Likert indicators (20 indicators total), using ordered probit models as in the basic HCM.

%% ============================================================================
%% SUBSECTION: Model 6 - ICLV
%% ============================================================================
\subsection{Model 6: Integrated Choice and Latent Variable}
\label{sec:iclv}

The Integrated Choice and Latent Variable (ICLV) model addresses the attenuation bias problem inherent in two-stage HCM estimation by simultaneously estimating all model components.

\subsubsection{Simultaneous Estimation}

Unlike two-stage HCM, ICLV estimates the structural, measurement, and choice models jointly. The individual contribution to the likelihood is:
\begin{equation}
L_n = \int \underbrace{P(y_n \,|\, \boldsymbol{\eta})}_{\text{Choice likelihood}} \times \underbrace{\prod_{k=1}^{K} P(I_{nk} \,|\, \boldsymbol{\eta})}_{\text{Measurement likelihood}} \times \underbrace{f(\boldsymbol{\eta} \,|\, \mathbf{z}_n)}_{\text{Structural density}} \, d\boldsymbol{\eta}
\label{eq:iclv_likelihood}
\end{equation}

\subsubsection{Components}

\textbf{Choice Likelihood}: Standard logit probability conditional on latent variables:
\begin{equation}
P(y_n = j \,|\, \boldsymbol{\eta}) = \frac{\exp(V_{nj}(\boldsymbol{\eta}))}{\sum_k \exp(V_{nk}(\boldsymbol{\eta}))}
\end{equation}

\textbf{Measurement Likelihood}: Product of ordered probit probabilities across indicators:
\begin{equation}
\prod_{k} P(I_{nk} \,|\, \boldsymbol{\eta}) = \prod_{k} \left[ \Phi(\tau_{k,c} - \lambda_k \eta) - \Phi(\tau_{k,c-1} - \lambda_k \eta) \right]^{\mathbf{1}_{I_{nk}=c}}
\end{equation}

\textbf{Structural Density}: Multivariate normal for latent variables:
\begin{equation}
f(\boldsymbol{\eta} \,|\, \mathbf{z}_n) = \phi(\boldsymbol{\eta} \,|\, \boldsymbol{\Gamma} \mathbf{z}_n, \boldsymbol{\Omega})
\end{equation}

\subsubsection{Monte Carlo Integration}

The integral in Equation~\eqref{eq:iclv_likelihood} is approximated via Monte Carlo simulation:
\begin{equation}
\tilde{L}_n = \frac{1}{R} \sum_{r=1}^{R} P(y_n \,|\, \boldsymbol{\eta}^{(r)}) \times \prod_{k} P(I_{nk} \,|\, \boldsymbol{\eta}^{(r)})
\label{eq:iclv_mc}
\end{equation}
where $\boldsymbol{\eta}^{(r)} \sim f(\boldsymbol{\eta} \,|\, \mathbf{z}_n)$ are draws from the conditional latent variable distribution.

\subsubsection{Bias Elimination}

By integrating over the distribution of latent variables rather than conditioning on point estimates, ICLV properly accounts for measurement uncertainty. Our validation studies demonstrate that ICLV reduces bias from 15--35\% (two-stage) to approximately 2--5\% (simultaneous), representing a substantial improvement in estimation accuracy.

%% ============================================================================
%% SUBSECTION: Model Progression Rationale
%% ============================================================================
\subsection{Model Progression Rationale}
\label{sec:progression}

The six-model progression follows a deliberate logic, with each specification addressing specific limitations of its predecessor.

\subsubsection{Progression Logic}

\begin{enumerate}
\item \textbf{MNL Basic $\rightarrow$ MNL Demographics}: Addresses the unrealistic assumption of homogeneous preferences by allowing observed characteristics to explain taste variation
\item \textbf{MNL Demographics $\rightarrow$ MXL}: Captures preference heterogeneity that \textit{cannot} be explained by observed demographics
\item \textbf{MXL $\rightarrow$ HCM Basic}: Provides a \textit{theoretical explanation} for unobserved heterogeneity through psychological constructs
\item \textbf{HCM Basic $\rightarrow$ HCM Full}: Enriches the psychological framework with multiple constructs and domain-specific effects
\item \textbf{HCM Full $\rightarrow$ ICLV}: Eliminates the attenuation bias inherent in two-stage estimation
\end{enumerate}

\subsubsection{Summary Comparison}

Table~\ref{tab:model_summary} provides a comprehensive comparison of the six specifications.

\begin{table}[htbp]
\centering
\caption{Comprehensive Model Comparison}
\label{tab:model_summary}
\begin{threeparttable}
\begin{tabular}{lcccccc}
\toprule
Feature & MNL & MNL-D & MXL & HCM-B & HCM-F & ICLV \\
\midrule
Observable heterogeneity & & \checkmark & & & & \\
Unobserved heterogeneity & & & \checkmark & \checkmark & \checkmark & \checkmark \\
Latent variables & & & & \checkmark & \checkmark & \checkmark \\
Multiple LVs & & & & & \checkmark & \checkmark \\
Simultaneous estimation & & & & & & \checkmark \\
\addlinespace
Relaxes IIA & & & \checkmark & \checkmark & \checkmark & \checkmark \\
Unbiased LV estimates & N/A & N/A & N/A & & & \checkmark \\
\addlinespace
Parameters (choice) & 3 & 8 & 4 & 4 & 11 & 4--11 \\
Estimation method & MLE & MLE & SML & Two-stage & Two-stage & SML \\
\bottomrule
\end{tabular}
\begin{tablenotes}
\small
\item Note: MNL = MNL Basic; MNL-D = MNL Demographics; HCM-B = HCM Basic; HCM-F = HCM Full. \checkmark\ indicates feature is present. MLE = Maximum Likelihood; SML = Simulated Maximum Likelihood.
\end{tablenotes}
\end{threeparttable}
\end{table}

\subsubsection{Recommended Usage}

The choice among specifications depends on the research objectives:

\begin{itemize}
\item \textbf{Baseline comparison}: MNL Basic provides a benchmark for more complex models
\item \textbf{Policy targeting}: MNL Demographics identifies demographic segments with different preferences
\item \textbf{Heterogeneity quantification}: MXL measures the extent of unexplained preference variation
\item \textbf{Theoretical testing}: HCM tests hypotheses about psychological determinants of choice
\item \textbf{Publication-quality estimates}: ICLV provides unbiased latent variable effects for substantive interpretation
\end{itemize}

%% ============================================================================
%% SUBSECTION: Validation Approach
%% ============================================================================
\subsection{Validation Approach}
\label{sec:validation_approach}

Each model specification is validated using a matching data generating process (DGP) that ensures the simulated data conform exactly to the model's assumptions.

\subsubsection{Parameter Recovery Protocol}

For each model, we:
\begin{enumerate}
\item Specify true parameter values in a configuration file
\item Generate synthetic data using the model's exact DGP
\item Estimate the model on synthetic data
\item Compare estimated to true parameters
\end{enumerate}

\subsubsection{Validation Metrics}

We assess parameter recovery using:
\begin{itemize}
\item \textbf{Percentage bias}: $100 \times (\hat{\theta} - \theta^*) / |\theta^*|$
\item \textbf{95\% CI coverage}: Proportion of estimates where $\theta^* \in [\hat{\theta} \pm 1.96 \times \text{SE}]$
\item \textbf{Significance}: $|t| = |\hat{\theta}/\text{SE}| > 1.96$
\end{itemize}

\subsubsection{Expected Performance}

Well-specified models should achieve:
\begin{itemize}
\item Bias $< 10\%$ for all parameters
\item Coverage $\approx 95\%$ (acceptable range: 90--97\%)
\item All parameters significant at $\alpha = 0.05$
\end{itemize}

The ICLV model, by eliminating attenuation bias, achieves substantially better performance on latent variable coefficients compared to two-stage HCM alternatives.

%% ============================================================================
%% End of Discrete Choice Model Framework Section
%% ============================================================================

%%
%% Required packages in main document:
%%   \usepackage{amsmath}
%%   \usepackage{amssymb}
%%   \usepackage{booktabs}
%%   \usepackage{threeparttable}
%%   \usepackage{multirow}

\section{Discrete Choice Model Framework}
\label{sec:dcm_framework}

This section presents the discrete choice modeling framework employed in our empirical analysis. We implement a sequence of six model specifications, each building upon its predecessor to address increasingly sophisticated forms of preference heterogeneity. The progression from basic multinomial logit through integrated choice and latent variable models provides both methodological rigor and substantive insights into the determinants of choice behavior.

%% ============================================================================
%% SUBSECTION: Framework Overview
%% ============================================================================
\subsection{Framework Overview}
\label{sec:framework_overview}

Our modeling framework comprises six discrete choice specifications organized in a deliberate hierarchy of complexity. Each model addresses a specific limitation of its predecessor while maintaining consistency in core structural assumptions. Table~\ref{tab:model_hierarchy} summarizes this progression.

\begin{table}[htbp]
\centering
\caption{Model Hierarchy and Heterogeneity Treatment}
\label{tab:model_hierarchy}
\begin{threeparttable}
\begin{tabular}{llccl}
\toprule
Model & Type & $K$ & Heterogeneity & Key Innovation \\
\midrule
1. MNL Basic & Baseline & 3 & None & Foundation specification \\
2. MNL Demographics & Observable & 8 & Demographics & Taste-demographic interactions \\
3. MXL Basic & Random Coef. & 4 & Unobserved & Distributional heterogeneity \\
4. HCM Basic & Hybrid & 5+ & Latent (single) & Psychological constructs \\
5. HCM Full & Hybrid & 11+ & Latent (multiple) & Domain-specific effects \\
6. ICLV & Integrated & 10+ & Latent (simult.) & Bias elimination \\
\bottomrule
\end{tabular}
\begin{tablenotes}
\small
\item Note: $K$ = number of estimated parameters (excluding measurement model parameters for HCM/ICLV). MNL = Multinomial Logit; MXL = Mixed Logit; HCM = Hybrid Choice Model; ICLV = Integrated Choice and Latent Variable.
\end{tablenotes}
\end{threeparttable}
\end{table}

The framework is designed around a common choice context: individuals choosing among three service alternatives---two paid expedited options and one standard (free) option with longer processing time. Attributes include processing fee (in Turkish Lira) and processing duration (in weeks). This unified context enables direct comparison of model performance and parameter estimates across specifications.

\subsubsection{Data Structure}

Each model operates on panel data with the following structure:
\begin{itemize}
\item $N = 500$ individuals
\item $T = 10$ choice tasks per individual
\item $J = 3$ alternatives per choice task
\item Total observations: $N \times T = 5{,}000$ choices
\end{itemize}

Attributes are drawn from a stratified-factorial experimental design that ensures orthogonality between fee and duration, enabling precise identification of both effects.

\subsubsection{Common Notation}

Throughout this section, we employ the following notation:
\begin{itemize}
\item $V_{nj}$ = Systematic utility of alternative $j$ for individual $n$
\item $\beta_{\text{fee}}$ = Marginal utility of fee (expected negative)
\item $\beta_{\text{dur}}$ = Marginal utility of duration (expected negative)
\item $\text{ASC}_{\text{paid}}$ = Alternative-specific constant for paid options
\item $\eta$ = Latent variable (psychological construct)
\item $\lambda$ = Factor loading (measurement model)
\item $\tau$ = Threshold parameter (ordered probit)
\end{itemize}

%% ============================================================================
%% SUBSECTION: Model 1 - MNL Basic
%% ============================================================================
\subsection{Model 1: Multinomial Logit Basic}
\label{sec:mnl_basic}

The multinomial logit (MNL) basic model serves as the foundation of our analysis, providing a parsimonious baseline against which more complex specifications are compared.

\subsubsection{Model Specification}

Under the assumption of homogeneous preferences, all individuals share identical taste parameters. The systematic utility for alternative $j$ is:
\begin{equation}
V_{nj} = \text{ASC}_{\text{paid}} \cdot \mathbf{1}_{j \in \{1,2\}} + \beta_{\text{fee}} \cdot \text{fee}_j + \beta_{\text{dur}} \cdot \text{dur}_j
\label{eq:mnl_basic_utility}
\end{equation}
where $\mathbf{1}_{j \in \{1,2\}}$ is an indicator function equal to 1 for paid alternatives and 0 for the standard option (which serves as the reference category with ASC normalized to zero).

\subsubsection{Choice Probability}

Given IID Gumbel-distributed errors, the choice probability takes the standard logit form:
\begin{equation}
P_{nj} = \frac{\exp(V_{nj})}{\exp(V_{n1}) + \exp(V_{n2}) + \exp(V_{n3})}
\label{eq:mnl_basic_prob}
\end{equation}

\subsubsection{Parameters}

The model estimates three parameters:

\begin{table}[htbp]
\centering
\caption{MNL Basic Parameters}
\label{tab:mnl_basic_params}
\begin{tabular}{llc}
\toprule
Parameter & Interpretation & Expected Sign \\
\midrule
$\text{ASC}_{\text{paid}}$ & Baseline preference for expedited service & + \\
$\beta_{\text{fee}}$ & Marginal disutility of fee (per 10,000 TL) & $-$ \\
$\beta_{\text{dur}}$ & Marginal disutility of duration (per week) & $-$ \\
\bottomrule
\end{tabular}
\end{table}

\subsubsection{Limitations}

The MNL basic model assumes that all individuals have identical preferences---a strong assumption that is unlikely to hold empirically. Observable differences in age, income, and education are ignored, as are unobserved psychological factors that may systematically influence choice.

%% ============================================================================
%% SUBSECTION: Model 2 - MNL Demographics
%% ============================================================================
\subsection{Model 2: Multinomial Logit with Demographics}
\label{sec:mnl_demo}

The MNL demographics model relaxes the homogeneity assumption by allowing taste parameters to vary as a function of observed individual characteristics.

\subsubsection{Model Specification}

We specify individual-specific coefficients that depend on demographic variables:
\begin{align}
\beta_{\text{fee},n} &= \beta_{\text{fee}} + \beta_{\text{fee,age}} \cdot \text{age}_n^c + \beta_{\text{fee,edu}} \cdot \text{edu}_n^c + \beta_{\text{fee,inc}} \cdot \text{inc}_n^c \label{eq:fee_demo} \\
\beta_{\text{dur},n} &= \beta_{\text{dur}} + \beta_{\text{dur,edu}} \cdot \text{edu}_n^c + \beta_{\text{dur,inc}} \cdot \text{inc}_n^c \label{eq:dur_demo}
\end{align}
where superscript $c$ denotes centered and scaled demographic variables.

The utility specification becomes:
\begin{equation}
V_{nj} = \text{ASC}_{\text{paid}} \cdot \mathbf{1}_{j \in \{1,2\}} + \beta_{\text{fee},n} \cdot \text{fee}_j + \beta_{\text{dur},n} \cdot \text{dur}_j
\label{eq:mnl_demo_utility}
\end{equation}

\subsubsection{Asymmetric Interaction Structure}

A notable feature of this specification is the asymmetric treatment of demographic interactions:
\begin{itemize}
\item \textbf{Fee sensitivity}: Varies with age, education, \textit{and} income
\item \textbf{Duration sensitivity}: Varies with education and income only (not age)
\end{itemize}

This asymmetry reflects theoretical considerations:
\begin{itemize}
\item \textbf{Age $\rightarrow$ Fee}: Older individuals may have accumulated wealth, reducing fee sensitivity
\item \textbf{Education/Income $\rightarrow$ Fee}: Higher socioeconomic status reduces price sensitivity
\item \textbf{Education/Income $\rightarrow$ Duration}: Higher opportunity cost of time increases duration sensitivity
\item \textbf{Age $\not\rightarrow$ Duration}: Time valuation driven by opportunity cost, not lifecycle stage
\end{itemize}

\subsubsection{Parameters}

The model estimates eight parameters:

\begin{table}[htbp]
\centering
\caption{MNL Demographics Parameters}
\label{tab:mnl_demo_params}
\begin{tabular}{llcc}
\toprule
Parameter & Interpretation & Expected Sign & Applies To \\
\midrule
$\text{ASC}_{\text{paid}}$ & Baseline expedited preference & + & All \\
$\beta_{\text{fee}}$ & Base fee sensitivity & $-$ & All \\
$\beta_{\text{fee,age}}$ & Age effect on fee sensitivity & + & Fee \\
$\beta_{\text{fee,edu}}$ & Education effect on fee sensitivity & + & Fee \\
$\beta_{\text{fee,inc}}$ & Income effect on fee sensitivity & + & Fee \\
$\beta_{\text{dur}}$ & Base duration sensitivity & $-$ & All \\
$\beta_{\text{dur,edu}}$ & Education effect on duration sensitivity & $-$ & Duration \\
$\beta_{\text{dur,inc}}$ & Income effect on duration sensitivity & $-$ & Duration \\
\bottomrule
\end{tabular}
\end{table}

Positive interaction coefficients on fee indicate \textit{reduced} fee sensitivity (less negative effective $\beta_{\text{fee}}$). Negative interaction coefficients on duration indicate \textit{increased} duration sensitivity (more negative effective $\beta_{\text{dur}}$).

%% ============================================================================
%% SUBSECTION: Model 3 - Mixed Logit
%% ============================================================================
\subsection{Model 3: Mixed Logit Basic}
\label{sec:mxl_basic}

The mixed logit (MXL) model captures unobserved preference heterogeneity by allowing taste parameters to follow continuous distributions across the population.

\subsubsection{Model Specification}

We specify the fee coefficient as normally distributed:
\begin{equation}
\beta_{\text{fee},n} \sim N(\mu_{\text{fee}}, \sigma^2_{\text{fee}})
\label{eq:mxl_distribution}
\end{equation}

Equivalently, using the standard normal draw representation:
\begin{equation}
\beta_{\text{fee},n} = \mu_{\text{fee}} + \sigma_{\text{fee}} \cdot \nu_n, \quad \nu_n \sim N(0,1)
\label{eq:mxl_draw}
\end{equation}

The utility specification is:
\begin{equation}
V_{nj} = \text{ASC}_{\text{paid}} \cdot \mathbf{1}_{j \in \{1,2\}} + \beta_{\text{fee},n} \cdot \text{fee}_j + \beta_{\text{dur}} \cdot \text{dur}_j
\label{eq:mxl_utility}
\end{equation}

Note that $\beta_{\text{dur}}$ is specified as fixed (non-random) to ensure identification and parsimony.

\subsubsection{Choice Probability}

The unconditional choice probability requires integration over the mixing distribution:
\begin{equation}
P_{nj} = \int \frac{\exp(V_{nj}(\beta))}{\sum_k \exp(V_{nk}(\beta))} \cdot \phi\left(\frac{\beta - \mu}{\sigma}\right) \cdot \frac{1}{\sigma} \, d\beta
\label{eq:mxl_prob}
\end{equation}

This integral lacks a closed-form solution. We approximate it using simulated maximum likelihood with $R = 500$ Halton draws:
\begin{equation}
\tilde{P}_{nj} = \frac{1}{R} \sum_{r=1}^{R} \frac{\exp(V_{nj}(\beta^{(r)}_n))}{\sum_k \exp(V_{nk}(\beta^{(r)}_n))}
\label{eq:mxl_simulated}
\end{equation}

\subsubsection{Parameters}

The model estimates four parameters:

\begin{table}[htbp]
\centering
\caption{MXL Basic Parameters}
\label{tab:mxl_params}
\begin{tabular}{llc}
\toprule
Parameter & Interpretation & Expected Sign \\
\midrule
$\text{ASC}_{\text{paid}}$ & Baseline expedited preference & + \\
$\mu_{\text{fee}}$ & Mean fee coefficient & $-$ \\
$\sigma_{\text{fee}}$ & Standard deviation of fee coefficient & + \\
$\beta_{\text{dur}}$ & Duration coefficient (fixed) & $-$ \\
\bottomrule
\end{tabular}
\end{table}

\subsubsection{Interpretation of Heterogeneity}

The coefficient of variation $\text{CV} = \sigma_{\text{fee}}/|\mu_{\text{fee}}|$ provides a normalized measure of heterogeneity. Under the normal specification:
\begin{itemize}
\item If $\text{CV} = 0.4$, approximately 1\% of the population has positive $\beta_{\text{fee}}$ (counterintuitive)
\item Large $\sigma$ relative to $|\mu|$ may indicate the need for alternative distributions (e.g., lognormal, constrained normal)
\end{itemize}

%% ============================================================================
%% SUBSECTION: Model 4 - HCM Basic
%% ============================================================================
\subsection{Model 4: Hybrid Choice Model Basic}
\label{sec:hcm_basic}

The hybrid choice model (HCM) introduces latent psychological constructs as determinants of choice behavior. The basic HCM specification incorporates a single latent variable.

\subsubsection{Conceptual Framework}

HCM extends the standard choice model by positing that unobserved psychological attitudes---measured imperfectly through survey indicators---systematically influence taste parameters. The model comprises three interconnected components:

\begin{enumerate}
\item \textbf{Structural model}: Links demographics to latent variables
\item \textbf{Measurement model}: Relates latent variables to observed indicators
\item \textbf{Choice model}: Incorporates latent variables as taste shifters
\end{enumerate}

\subsubsection{Structural Model}

The latent variable (Blind Patriotism, $\eta_{\text{pb}}$) is determined by demographics:
\begin{equation}
\eta_{\text{pb},n} = \gamma_0 + \gamma_{\text{age}} \cdot (\text{age}_n - \bar{\text{age}}) + \zeta_n, \quad \zeta_n \sim N(0, \sigma^2_\zeta)
\label{eq:hcm_structural}
\end{equation}

This specification reflects the hypothesis that patriotic attitudes vary systematically with age, with a residual component capturing individual-specific variation.

\subsubsection{Measurement Model}

The latent variable is measured through five Likert-scale indicators ($I_1, \ldots, I_5$), each following an ordered probit specification:
\begin{equation}
P(I_{nk} = c \,|\, \eta_n) = \Phi(\tau_{k,c} - \lambda_k \eta_n) - \Phi(\tau_{k,c-1} - \lambda_k \eta_n)
\label{eq:hcm_measurement}
\end{equation}
where:
\begin{itemize}
\item $\lambda_k$ = factor loading for indicator $k$ (first loading fixed to 1 for identification)
\item $\tau_{k,c}$ = threshold for category $c$ of indicator $k$
\item $\Phi(\cdot)$ = standard normal CDF
\end{itemize}

\subsubsection{Choice Model}

The latent variable enters the choice model as a taste shifter:
\begin{align}
\beta_{\text{fee},n} &= \beta_{\text{fee}} + \lambda_{\text{fee,pb}} \cdot \eta_{\text{pb},n} \label{eq:hcm_fee} \\
V_{nj} &= \text{ASC}_{\text{paid}} \cdot \mathbf{1}_{j \in \{1,2\}} + \beta_{\text{fee},n} \cdot \text{fee}_j + \beta_{\text{dur}} \cdot \text{dur}_j \label{eq:hcm_utility}
\end{align}

Negative $\lambda_{\text{fee,pb}}$ indicates that higher Blind Patriotism reduces fee sensitivity (individuals with stronger patriotic attitudes are more willing to pay for government services).

\subsubsection{Estimation Challenge: Attenuation Bias}

When estimated via two-stage procedures (factor analysis followed by choice model estimation), HCM suffers from attenuation bias. The estimated latent variable effect is systematically underestimated by 15--30\% due to measurement error in the first-stage latent scores.

%% ============================================================================
%% SUBSECTION: Model 5 - HCM Full
%% ============================================================================
\subsection{Model 5: Hybrid Choice Model Full}
\label{sec:hcm_full}

The full HCM specification extends the basic model to incorporate four latent psychological constructs with domain-specific effects.

\subsubsection{Four Latent Variables}

We model four attitudinal constructs:

\begin{enumerate}
\item \textbf{Blind Patriotism ($\eta_{\text{pb}}$)}: Uncritical, unwavering support for one's country
\item \textbf{Constructive Patriotism ($\eta_{\text{pc}}$)}: Critical, improvement-oriented attachment
\item \textbf{Daily Life Secularism ($\eta_{\text{sdl}}$)}: Preference for separation of religion in daily practices
\item \textbf{Faith \& Prayer Secularism ($\eta_{\text{sfp}}$)}: Preference for separation in religious matters
\end{enumerate}

\subsubsection{Structural Models}

Each latent variable has its own structural equation:
\begin{align}
\eta_{\text{pb},n} &= \gamma_{\text{pb,age}} \cdot \text{age}_n^c + \gamma_{\text{pb,inc}} \cdot \text{inc}_n^c + \zeta_{\text{pb},n} \label{eq:hcm_full_pb} \\
\eta_{\text{pc},n} &= \gamma_{\text{pc,edu}} \cdot \text{edu}_n^c + \zeta_{\text{pc},n} \label{eq:hcm_full_pc} \\
\eta_{\text{sdl},n} &= \gamma_{\text{sdl,edu}} \cdot \text{edu}_n^c + \gamma_{\text{sdl,inc}} \cdot \text{inc}_n^c + \zeta_{\text{sdl},n} \label{eq:hcm_full_sdl} \\
\eta_{\text{sfp},n} &= \gamma_{\text{sfp,edu}} \cdot \text{edu}_n^c + \zeta_{\text{sfp},n} \label{eq:hcm_full_sfp}
\end{align}

\subsubsection{Domain Separation Hypothesis}

A key theoretical contribution is the domain separation hypothesis: different psychological constructs affect different behavioral domains:

\begin{itemize}
\item \textbf{Patriotism $\rightarrow$ Fee sensitivity}: Patriotic attitudes affect willingness to pay for government services
\item \textbf{Secularism $\rightarrow$ Duration sensitivity}: Secular attitudes affect time valuation and patience
\end{itemize}

This yields the choice model specification:
\begin{align}
\beta_{\text{fee},n} &= \beta_{\text{fee}} + \lambda_{\text{pb}} \eta_{\text{pb},n} + \lambda_{\text{pc}} \eta_{\text{pc},n} + \lambda_{\text{sdl}} \eta_{\text{sdl},n} + \lambda_{\text{sfp}} \eta_{\text{sfp},n} \\
\beta_{\text{dur},n} &= \beta_{\text{dur}} + \lambda'_{\text{pb}} \eta_{\text{pb},n} + \lambda'_{\text{pc}} \eta_{\text{pc},n} + \lambda'_{\text{sdl}} \eta_{\text{sdl},n} + \lambda'_{\text{sfp}} \eta_{\text{sfp},n}
\end{align}

\subsubsection{Measurement Structure}

Each latent variable is measured by five Likert indicators (20 indicators total), using ordered probit models as in the basic HCM.

%% ============================================================================
%% SUBSECTION: Model 6 - ICLV
%% ============================================================================
\subsection{Model 6: Integrated Choice and Latent Variable}
\label{sec:iclv}

The Integrated Choice and Latent Variable (ICLV) model addresses the attenuation bias problem inherent in two-stage HCM estimation by simultaneously estimating all model components.

\subsubsection{Simultaneous Estimation}

Unlike two-stage HCM, ICLV estimates the structural, measurement, and choice models jointly. The individual contribution to the likelihood is:
\begin{equation}
L_n = \int \underbrace{P(y_n \,|\, \boldsymbol{\eta})}_{\text{Choice likelihood}} \times \underbrace{\prod_{k=1}^{K} P(I_{nk} \,|\, \boldsymbol{\eta})}_{\text{Measurement likelihood}} \times \underbrace{f(\boldsymbol{\eta} \,|\, \mathbf{z}_n)}_{\text{Structural density}} \, d\boldsymbol{\eta}
\label{eq:iclv_likelihood}
\end{equation}

\subsubsection{Components}

\textbf{Choice Likelihood}: Standard logit probability conditional on latent variables:
\begin{equation}
P(y_n = j \,|\, \boldsymbol{\eta}) = \frac{\exp(V_{nj}(\boldsymbol{\eta}))}{\sum_k \exp(V_{nk}(\boldsymbol{\eta}))}
\end{equation}

\textbf{Measurement Likelihood}: Product of ordered probit probabilities across indicators:
\begin{equation}
\prod_{k} P(I_{nk} \,|\, \boldsymbol{\eta}) = \prod_{k} \left[ \Phi(\tau_{k,c} - \lambda_k \eta) - \Phi(\tau_{k,c-1} - \lambda_k \eta) \right]^{\mathbf{1}_{I_{nk}=c}}
\end{equation}

\textbf{Structural Density}: Multivariate normal for latent variables:
\begin{equation}
f(\boldsymbol{\eta} \,|\, \mathbf{z}_n) = \phi(\boldsymbol{\eta} \,|\, \boldsymbol{\Gamma} \mathbf{z}_n, \boldsymbol{\Omega})
\end{equation}

\subsubsection{Monte Carlo Integration}

The integral in Equation~\eqref{eq:iclv_likelihood} is approximated via Monte Carlo simulation:
\begin{equation}
\tilde{L}_n = \frac{1}{R} \sum_{r=1}^{R} P(y_n \,|\, \boldsymbol{\eta}^{(r)}) \times \prod_{k} P(I_{nk} \,|\, \boldsymbol{\eta}^{(r)})
\label{eq:iclv_mc}
\end{equation}
where $\boldsymbol{\eta}^{(r)} \sim f(\boldsymbol{\eta} \,|\, \mathbf{z}_n)$ are draws from the conditional latent variable distribution.

\subsubsection{Bias Elimination}

By integrating over the distribution of latent variables rather than conditioning on point estimates, ICLV properly accounts for measurement uncertainty. Our validation studies demonstrate that ICLV reduces bias from 15--35\% (two-stage) to approximately 2--5\% (simultaneous), representing a substantial improvement in estimation accuracy.

%% ============================================================================
%% SUBSECTION: Model Progression Rationale
%% ============================================================================
\subsection{Model Progression Rationale}
\label{sec:progression}

The six-model progression follows a deliberate logic, with each specification addressing specific limitations of its predecessor.

\subsubsection{Progression Logic}

\begin{enumerate}
\item \textbf{MNL Basic $\rightarrow$ MNL Demographics}: Addresses the unrealistic assumption of homogeneous preferences by allowing observed characteristics to explain taste variation
\item \textbf{MNL Demographics $\rightarrow$ MXL}: Captures preference heterogeneity that \textit{cannot} be explained by observed demographics
\item \textbf{MXL $\rightarrow$ HCM Basic}: Provides a \textit{theoretical explanation} for unobserved heterogeneity through psychological constructs
\item \textbf{HCM Basic $\rightarrow$ HCM Full}: Enriches the psychological framework with multiple constructs and domain-specific effects
\item \textbf{HCM Full $\rightarrow$ ICLV}: Eliminates the attenuation bias inherent in two-stage estimation
\end{enumerate}

\subsubsection{Summary Comparison}

Table~\ref{tab:model_summary} provides a comprehensive comparison of the six specifications.

\begin{table}[htbp]
\centering
\caption{Comprehensive Model Comparison}
\label{tab:model_summary}
\begin{threeparttable}
\begin{tabular}{lcccccc}
\toprule
Feature & MNL & MNL-D & MXL & HCM-B & HCM-F & ICLV \\
\midrule
Observable heterogeneity & & \checkmark & & & & \\
Unobserved heterogeneity & & & \checkmark & \checkmark & \checkmark & \checkmark \\
Latent variables & & & & \checkmark & \checkmark & \checkmark \\
Multiple LVs & & & & & \checkmark & \checkmark \\
Simultaneous estimation & & & & & & \checkmark \\
\addlinespace
Relaxes IIA & & & \checkmark & \checkmark & \checkmark & \checkmark \\
Unbiased LV estimates & N/A & N/A & N/A & & & \checkmark \\
\addlinespace
Parameters (choice) & 3 & 8 & 4 & 4 & 11 & 4--11 \\
Estimation method & MLE & MLE & SML & Two-stage & Two-stage & SML \\
\bottomrule
\end{tabular}
\begin{tablenotes}
\small
\item Note: MNL = MNL Basic; MNL-D = MNL Demographics; HCM-B = HCM Basic; HCM-F = HCM Full. \checkmark\ indicates feature is present. MLE = Maximum Likelihood; SML = Simulated Maximum Likelihood.
\end{tablenotes}
\end{threeparttable}
\end{table}

\subsubsection{Recommended Usage}

The choice among specifications depends on the research objectives:

\begin{itemize}
\item \textbf{Baseline comparison}: MNL Basic provides a benchmark for more complex models
\item \textbf{Policy targeting}: MNL Demographics identifies demographic segments with different preferences
\item \textbf{Heterogeneity quantification}: MXL measures the extent of unexplained preference variation
\item \textbf{Theoretical testing}: HCM tests hypotheses about psychological determinants of choice
\item \textbf{Publication-quality estimates}: ICLV provides unbiased latent variable effects for substantive interpretation
\end{itemize}

%% ============================================================================
%% SUBSECTION: Validation Approach
%% ============================================================================
\subsection{Validation Approach}
\label{sec:validation_approach}

Each model specification is validated using a matching data generating process (DGP) that ensures the simulated data conform exactly to the model's assumptions.

\subsubsection{Parameter Recovery Protocol}

For each model, we:
\begin{enumerate}
\item Specify true parameter values in a configuration file
\item Generate synthetic data using the model's exact DGP
\item Estimate the model on synthetic data
\item Compare estimated to true parameters
\end{enumerate}

\subsubsection{Validation Metrics}

We assess parameter recovery using:
\begin{itemize}
\item \textbf{Percentage bias}: $100 \times (\hat{\theta} - \theta^*) / |\theta^*|$
\item \textbf{95\% CI coverage}: Proportion of estimates where $\theta^* \in [\hat{\theta} \pm 1.96 \times \text{SE}]$
\item \textbf{Significance}: $|t| = |\hat{\theta}/\text{SE}| > 1.96$
\end{itemize}

\subsubsection{Expected Performance}

Well-specified models should achieve:
\begin{itemize}
\item Bias $< 10\%$ for all parameters
\item Coverage $\approx 95\%$ (acceptable range: 90--97\%)
\item All parameters significant at $\alpha = 0.05$
\end{itemize}

The ICLV model, by eliminating attenuation bias, achieves substantially better performance on latent variable coefficients compared to two-stage HCM alternatives.

%% ============================================================================
%% End of Discrete Choice Model Framework Section
%% ============================================================================
